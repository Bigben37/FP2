\section{Physikalische Grundlagen}

3-4 Seiten, nur was für Auswertung relevant


\subsection{Hyperfeinstruktur und Zeeman-Effekt}
1 Termschema

\subsection{Optisches Pumpen}


\subsection{Die Laserdiode}
Zum optischen Pumpen wird im Versuch eine Laserdiode verwendet,
weil Laserdioden linear polarisiertes Licht in einem schmalen Wellenlängenbereich liefern.
Durch den Laserstrom und die Lasertemperatur kann die Wellenlänge des Lasers gezielt
beeinflusst werden, so dass selektiv einzelne Hyperfein-Übergänge angeregt werden können.
Für den Betrieb des Lasers ist ein Minimalstrom notwendig (die \emph{Laserschwelle}).
Wird dieser Strom überschritten, ändern sich Ausgangsleistung und Laserfrequenz
in erster Näherung linear mit dem Laserstrom.
Es treten allerdings bei der Durchstimmung immer wieder Modensprünge auf,
wenn sich die Zahl der stehenden Wellen im Resonator ändert.
Bei den Messungen muss daher darauf geachtet werden, dass keine Modensprünge auftreten,
um eine lineare Durchstimmung des Lasers zu erzielen.
Die Linearität des Lasers kann mit einem Fabry-Pérot-Interferometer überprüft werden.


\subsection{Das Fabry-Pérot-Interferometer}
Ein Fabry-Pérot-Interferometer (Etalon) besteht aus zwei halbtransparenten, reflektierenden Flächen
im Strahlengang.
Eine Lichtwelle kann transmittiert werden, wenn sie zwischen den Spiegeln konstruktiv mit sich selbst interferiert.
Über die Bedingung für konstruktive Interferenz kann man den \emph{freien Spektralbereich} berechnen,
der den Abstand zwischen zwei Transmissionsmaxima beschreibt.
Bei dem im Versuch verwendeten Etalon beträgt er $\Delta \nu_\text{FSR} = 9924 \pm 30\,\text{MHz}$.


\subsection{Spinpräzession des Rubidiumensembles im äußeren Magnetfeld}
Der Mechanismus der Spinpräzession wird in einem Versuchsteil ausgenutzt,
um die Stärke eines Magnetfelds sehr genau zu bestimmen.
Spinpräzession tritt auf, wenn ein Ensemble von Atomen im Magnetfeld optisch gepumpt,
das Ensemble also polarisiert wird, und anschließend eine Komponente
des Magnetfelds schnell abgeschaltet wird (im Experiment die Komponente in Strahlrichtung).
Die Spins der Atome führen dann eine Präzessionsbewegung um das verbleibende Magnetfeld $B$ aus,
im Experiment ist das die Vertikalkomponente.
Die Präzessionsfrequenz $f$ beträgt \cite{staatsex}
\begin{equation}
\label{eq:gr:spinpräz}
  f=\frac{g_\text{F} \cdot \mu_\text{B}}{h} \cdot B=: \alpha \cdot B
\end{equation}
Hier sind $\mu_\text{B}$ das Bohrsche Magneton, $h$ das Plancksche Wirkungsquantum und
$g_\text{F}$ der Landé-Faktor des Isotops.
Man erhält mit $g_\text{F}$(\rb{85})=1/3 und $g_\text{F}$(\rb{87})=1/2 \cite{staatsex} für 
die Proportionalitätskonstante $\alpha$ die Werte
\begin{equation}
  \alpha_{85}=4.665\,\text{kHz } \text{\textmu T}^{-1} \qquad \text{und} \qquad	\alpha_{87}=6.998\,\text{kHz } \text{\textmu T}^{-1} \ \, .
\end{equation}



\subsection{Orientierungsprozesse im Rubidiumensemble}
In diesem Abschnitt wird ein kurzer Überblick über die mathematische Beschreibung der Orientierungsprozesse
(Polarisation und Relaxation) im Rubidiumgas gegeben und es werden die Zusammenhänge aufgeführt, die für die
Auswertung der Messungen zur Relaxationszeit notwendig sind.
Die ausführliche Herleitung der Formeln wird in \cite{staatsex} gezeigt.

Betrachtet man ein Ensemble von Rubidiumatomen, das im Magnetfeld durch zirkular polarisiertes Licht gepumpt wird,
wird die Zunahme der Differenz der Besetzungszahlen~$\left(\frac{\difd n}{\difd t}\right)_{\text{Pump}}$
durch die folgende Differenzialgleichung beschrieben:
\begin{equation}
  \left(\frac{\difd n}{\difd t}\right)_{\text{Pump}}=\frac{N-n}{T_\text{P}}
\end{equation}
$N$ ist die Anzahl der Atome im Ensemble, $n$ die Differenz der Besetzungszahlen in dem Zweiniveausystem
und $T_\text{P}$ die charakteristische Zeit für den Pumpvorgang, die \emph{Pumpzeit}.

Relaxation, also Verlust der Polarisation, ist durch verschiedene Mechanismen möglich:
Durch Wechselwirkung der Atome mit der Wand der Messzelle, durch Stöße mit dem Puffergas
oder durch Spinaustausch zwischen den Rubidiumatomen.
Der Relaxationsvorgang wird durch folgende Differenzialgleichung beschrieben:
\begin{equation}
  \left(\frac{\difd n}{\difd t}\right)_{\text{Relax}}=-\frac{n}{T_\text{R}}
\end{equation}
$T_\text{R}$ ist die \emph{Relaxationszeit}.

Die Summe von Polarisation und Relaxation beschreibt den Orientierungsprozess im Rubidiumgas mit
der Orientierungszeit $\tau$:
\begin{equation}
\label{eq:orientierungszeit}
  \left(\frac{\difd n}{\difd t}\right)_{\text{Orient}}
  =\left(\frac{\difd n}{\difd t}\right)_{\text{Pump}} + \left(\frac{\difd n}{\difd t}\right)_{\text{Relax}}
  =\frac{N}{T_\text{P}}-n \left( \frac{1}{T_\text{P}} + \frac{1}{T_\text{R}}\right)
  =:\frac{N}{T_\text{P}}- \frac{n}{\tau}
\end{equation}

Die Lösung dieser Differenzialgleichung für $n(t)$ ist eine exponentielle Änderung mit der Zeitkonstante $\tau$:
\begin{equation}
\label{eq:expabhorient}
   n(t) \sim e^{-\frac{t}{\tau}}
\end{equation}


