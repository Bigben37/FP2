\section{Physikalische Grundlagen}

3-4 Seiten, nur was für Auswertung relevant

1 Termschema

Diodenlaser, Modensprünge. Nix mit pn..

Zeeman, Hyperfein

Etalon

\subsection{Orientierungsprozesse im Rubidiumensemble}
In diesem Abschnitt wird ein kurzer Überblick über die mathematische Beschreibung der Orientierungsprozesse
(Polarisation und Relaxation) im Rubidiumgas gegeben und es werden die Zusammenhänge aufgeführt, die für die
Auswertung der Messungen zur Relaxationszeit notwendig sind.
Die ausführliche Herleitung der Formeln ist in \cite{staatsex} gezeigt.

Bei einem Ensemble von Rubidiumatomen, das im Magnetfeld durch zirkular polarisiertes Licht gepumpt wird,
wird die zeitliche Änderung der Differenz der Besetzungszahlen~$\left(\frac{\difd n}{\difd t}\right)_{\text{Pump}}$
durch die folgende Differenzialgleichung beschrieben:
\begin{equation}
  \left(\frac{\difd n}{\difd t}\right)_{\text{Pump}}=\frac{N-n}{T_\text{P}}
\end{equation}
$N$ ist die Anzahl der Atome im Ensemble, $n$ die Differenz der Besetzungszahlen in dem Zweiniveausystem
und $T_\text{P}$ die charakteristische Zeit für den Pumpvorgang, die \emph{Pumpzeit}.

Der Relaxationsvorgang wird durch folgende Differenzialgleichung beschrieben:
\begin{equation}
  \left(\frac{\difd n}{\difd t}\right)_{\text{Relax}}=-\frac{n}{T_\text{R}}
\end{equation}
$T_\text{R}$ ist die \emph{Relaxationszeit}.

Die Summe von Polarisation und Relaxation beschreibt den Orientierungsprozess im Rubidiumgas mit
der Orientierungszeit $\tau$:
\begin{equation}
\label{eq:orientierungszeit}
  \left(\frac{\difd n}{\difd t}\right)_{\text{Orient}}
  =\left(\frac{\difd n}{\difd t}\right)_{\text{Pump}} + \left(\frac{\difd n}{\difd t}\right)_{\text{Relax}}
  =\frac{N}{T_\text{P}}-n \left( \frac{1}{T_\text{P}} + \frac{1}{T_\text{R}}\right)
  =:\frac{N}{T_\text{P}}- \frac{n}{\tau}
\end{equation}

Die Lösung dieser Differenzialgleichung für $n(t)$ ist eine exponentielle Änderung mit der Zeitkonstante $\tau$:
\begin{equation}
   n(t) \sim e^{-\frac{t}{\tau}}
\end{equation}


