\section{Versuchsaufbau}
\autoref{img:aufbau} zeigt den Aufbau, der für die Durchführung der Versuche
verwendet wird.
Auf einer optischen Bank befindet sich eine \emph{Laserdiode} (mit der Wellenlänge
795\,nm zur Anregung der D$_1$-Linie von Rubidium).
Die Temperatur der Diode kann mit einem \emph{Peltierelement} eingestellt werden,
die Stromversorgung erfolgt durch ein speziellen Netzteil.
Der Laserstrahl wird von einer \emph{Linse} kollimiert und durchläuft verschiedene
optische Bauteile, die (nicht alle gleichzeitig) in den einzelnen Versuchsteilen
verwendet werden:

Eine drehbare Scheibe (\emph{Chopper}), in der sich Löcher befinden, kann von
einem Elektromotor auf gewünschte Rotationsgeschwindigkeit eingestellt werden und unterbricht periodisch
den Laserstrahl.
Um die Fre\-quenz\-eich\-ung der Lasers durchzuführen, kann ein \emph{Etalon}
(mit 9924\,$\pm$\,30\,MHz freiem Spektralbereich) in den Strahlengang gebracht werden.
Ein \emph{$\lambda$/4-Plättchen} erzeugt aus dem linear polarisierten Laserlicht zirkular polarisiertes Licht.
Ein \emph{Linearpolarisator} kann als Analysator verwendet werden, um den Polarisationszustand
des Strahls zu überprüfen. Verschiedene \emph{Neutraldichtefilter} können zur Verringerung der Laserlichtintensität
benutzt werden.

Anschließend fällt das Laserlicht durch eine mit Rubidium gefüllte Glaskugel.
Um den Dampfdruck des Rubidiums zu erhöhen, kann diese Zelle mit einem Föhn geheizt werden.
Parallel zum Strahlverlauf kann mit den \emph{Helmholtzspulenpaaren 1}, \emph{2} und \emph{3}
ein Magnetfeld in der Zelle erzeugt werden;
zusätzlich mit einer kleinen Spule (\emph{Spule 5}) direkt auf der Glaskugel.
Ein vertikales Magnetfeld wird mit \emph{Spule 4} erzeugt.
Ein senkrecht zum Strahl verlaufendes horizontales Magnetfeld bleibt unkompensiert.
Alle Spulen werden von Frequenzgeneratoren und Konstantstromquellen versorgt.
Um die Stärke des Magnetfeldes bei einem bestimmten Strom zu berechnen, können die Werte in \autoref{tab:inductorItoB} benutzt werden.
\begin{table}[H]
\caption{Faktoren $c_n$ für die Umrechnung von durchlaufendem Strom in Stärke des erzeugten Magnetfeldes der verschiedenen Spulen, 
		 entnommen aus \cite{manual}.}
\begin{center}
\begin{tabular}{|c|c|c|}
    \hline
    Spule $n$	& $c_n$ / $\frac{\text{T}}{\text{A}}$	& $s_{c_n}$ / $\frac{\text{T}}{\text{A}}$	\\ \hline
    1			& $7.99 \cdot 10^{-4}$					& $0.01 \cdot 10^{-4}$						\\ \hline
    4			& $4.76 \cdot 10^{-4}$					& $0.01 \cdot 10^{-4}$						\\ \hline
\end{tabular}
\end{center}
\label{tab:inductorItoB}
\end{table}

An der Zelle befindet sich ein \emph{RF-Sender}, mit dem ein Signal im Radiofrequenzbereich eingestrahlt werden kann.

Eine zweite \emph{Linse} fokussiert den Laserstrahl auf eine \emph{Photodiode}, deren Signal verstärkt und auf
einem Oszilloskop angezeigt wird.
Auch die Signale der Frequenzgeneratoren können mit dem Oszilloskop betrachtet werden.


\begin{landscape}
%\thispagestyle{plain}

\begin{figure}[H]
        %\centering
        \def\svgwidth{1.4\textwidth}
        \input{../img/aufbau.pdf_tex}
        \caption{Aufbau des Experiments: Strahlengang auf der optischen Bank
        und optische und elektronische Bauteile.
        Nicht alle gezeigten Komponenten werden in den einzelnen
        Versuchsteilen verwendet.}
        \label{img:aufbau}
\end{figure}

\end{landscape}
