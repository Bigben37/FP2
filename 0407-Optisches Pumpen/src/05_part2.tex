\section{Spektroskopie der Hyperfeinstruktur von Rubidium}
\subsection{Durchführung}
Bei der Messung des Hyperfein-Absorptionsspektrums befinden sich die beiden Linsen, die Rubidiumzelle und
das $\lambda$/4-Plättchen im Strahlengang.
Mit dem Linearpolarisator (im Strahlengang nach dem $\lambda$/4-Plättchen) wurde vor der Messung die
korrekte Win\-kel\-ein\-stel\-lung des $\lambda$/4-Plättchens gefunden,
indem die Schwankungen der transmittierten Intensität
bei Drehung des Polarisators minimiert wurden.

Der Konstantaneil des Laserstroms, Modulationsamplitude und -frequenz
sind wie bei der Messung der Zeitabhängigkeit der Laserfrequenz.
Äußere Magnetfelder bleiben unkompensiert, weil die Zeeman-Aufspaltung der Hyperfeinstruktur im Erdmagnetfeld
mit der Linienbreite der Laserdiode nicht auflösbar ist.

Die Messung wird auf der steigenden und der fallenden Flanke der Modulationsspannung durchgeführt.


\subsection{Auswertung}
\subsubsection*{Frequenzeichung}
\subsubsection*{Hyperfeinstruktur-Übergänge}
\subsubsection*{Berechnung des Spektrums}
\subsubsection*{Vergleich mit den Literaturwerten}
Gerade sollte Steigung 1 und Achsenabschnitt 0 Ghz haben
