\section{Spektroskopie der Hyperfeinstruktur von Rubidium}
\subsection{Durchführung}
Bei der Messung des Hyperfein-Absorptionsspektrums befinden sich die beiden Linsen und
die Rubidiumzelle im Strahlengang.
Der Konstantanteil des Laserstroms, Modulationsamplitude und -frequenz
sind wie bei der Messung der Zeitabhängigkeit der Laserfrequenz (\autoref{sect:durchführung}).
Äußere Magnetfelder bleiben unkompensiert, weil die Zeeman-Aufspaltung der Hyperfeinstruktur im Erdmagnetfeld
mit der Linienbreite der Laserdiode nicht auflösbar ist.
Die Messung wird auf der steigenden und der fallenden Flanke der Modulationsspannung durchgeführt und
während der Messung wird die Zelle mit dem Föhn geheizt.


\subsection{Auswertung}
Da in dieses Teil der Auswertung häufiger Kurvenanpassungen mit einer Gauß-Funktion durchgeführt werden, wird folgende Konvention eingeführt:
\begin{equation}
    \label{eq:convention:gauss}
    \gaus(x; A, \mu, \sigma) = A \cdot e^{-\frac{1}{2} \left( \frac{x-\mu}{\sigma} \right)^2}
\end{equation}
\subsubsection*{Frequenzkalibrierung}
Das Etalonspektrum und die Spannung der Lasermodulation sind in \autoref{img:etalon:fit} abgebildet. 
Die Peaks des Etalonspektrums werden mit Gauß-Kurven und einem linearen Untergrund gefittet. 
\begin{equation}
    U_\text{ph}(t) = a + b \cdot t + \gaus(t; A, \mu, \sigma)
\end{equation}
Des Weiteren wird die Spannung für die Lasermodulation mit einer Geraden gefittet.
\begin{equation}
    U_\text{L} = a + \cdot t
\end{equation}
\begin{figure}[H]
\begin{center}
  \includegraphics[width=\textwidth]{../img/part2/up-etalon_zoom_fit.pdf}
  \caption{caption.}
  \label{img:etalon:fit}
\end{center}
\end{figure}
TODO: Farbcode ändern \\
Gerade von $U_\text{L}$ wird nur betrachtet, um zu kontrollieren, ob die Spannung auch gerade ansteigt. \\
TODO: Tabelle mit Peakdaten \\
TODO: Mehr peaks fitten, wenn möglich

Aus dem freien Spektralbereich des Etalons $\Delta \nu_\text{FSR} = 9924 \pm 30\,\text{MHz}$ lässt sich nun die Differenz der Laserfrequenz bei den verschiedenen 
Peaks bestimmen. Der erste Peak wird als Referenzpeak festgelegt. Der Frequenzabstand $\nu_i$ zwischen erstem und $i$-ten Peak lässt sich nun 
folgendermaßen berechnen\footnote{Gedankenspiel zur Fehlerrechnung: Interpretiert man $2 \cdot a$ als $a + a$, so ist der Fehler im ersten Fall $2 \cdot s_a$, im zweiten 
allerdings $\sqrt{2} \cdot s_a$, wenn man die Selbstkorrelation von $a$ nicht berücksichtigt. Mit $\cor(a, a) = 1$ 
erhält man $\sqrt{s_a^2 + s_a^2 + 2 \cdot s_a \cdot s_a \cdot \cor(a, a)} = 2 \cdot s_a$.}: 
\begin{equation}
    \Delta \nu_i = \Delta \nu_\text{FSR} \cdot i, \qquad s_{\Delta \nu_i} = \sqrt{i} \cdot s_{\Delta \nu_\text{FSR}} 
\end{equation}
Die Frequenzabstände werden nun gegen die Maximumspositionen (also die Erwartungswerte der Gauß-Funktionen) der Etalonpeaks aufgetragen ( \autoref{img:etalon:calibration}).
\begin{figure}[H]
\begin{center}
  \includegraphics[width=\textwidth]{../img/part2/up-etalon_zoom-etalon_calibration.pdf}
  \caption{caption.}
  \label{img:etalon:calibration}
\end{center}
\end{figure}
Aus dem Fit mit einer Geraden 
\begin{equation}
    \Delta \nu(t) = a + r \cdot t  %TODO Parameter in Plot ändern
\end{equation}
lässt sich nun die Scanrate $r$ bestimmen. Man erhält %TODO Bessere Variable für Scanrate
\begin{equation}
    r = (9.40 \pm 0.02)\,\frac{\text{GHz}}{ms}\ \, .
\end{equation}

\subsubsection*{Hyperfeinstruktur-Übergänge}
nur 6 peaks erkennbar, da 2 nicht aufgelößt werden können \\
fit mit überlagerten Gauß \\
tabelle mit peak-positionen
\begin{figure}[H]
\begin{center}
  \includegraphics[width=\textwidth]{../img/part2/up-hfs_zoom_fit.pdf}  %TODO Better start params
  \caption{caption.}
  \label{img:hfs:fit:up}
\end{center}
\end{figure}

\subsubsection*{Berechnung des Spektrums}
festlegung referenzpeak \\
berechnung des spektrums 

\subsubsection*{Vergleich mit den Literaturwerten}
Tabelle mit Spektrumswerten (Literaturwert, aufsteigend, absteigend (eine Tabelle))
Gerade sollte Steigung 1 und Achsenabschnitt 0 Ghz haben

\begin{figure}[H]
\begin{center}
  \includegraphics[width=\textwidth]{../img/part2/up-spectrum.pdf}
  \caption{caption.}
  \label{img:hfs:spectrum:up}
\end{center}
\end{figure}

\begin{figure}[H]
\begin{center}
  \includegraphics[width=\textwidth]{../img/part2/down-spectrum.pdf}
  \caption{caption.}
  \label{img:hfs:spectrum:down}
\end{center}
\end{figure}
