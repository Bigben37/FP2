\section{Charakterisierung des Messaufbaus}
\subsection{Durchführung}
\subsubsection*{P-I-Kennlinie der Laserdiode}
Für die erste Messung werden nur die beiden Linsen im Strahlengang justiert, alle anderen Elemente sind entfernt. 
Das Peltierelement wird auf 34.0$^\circ$C eingestellt und die Spannung an der
Photodiode bei verschiedenen Laserdiodenströmen aufgezeichnet.
%TODO 34grad?

\subsubsection*{Temperaturabhängigkeit der Laserfrequenz}
Zur Bestimmung der Temperaturabhängigkeit der
Laserfrequenz wird das Etalon in den Strahlengang eingesetzt und der Diodenstrom mit einem Dreieckssignal
moduliert. Der Konstantanteil am Strom beträgt XXX, die Modulationsamplitude XXX.
Modulationssignal und Photodiodensignal werden mit dem Oszilloskop gemessen und die Wanderung eines Etalonpeaks
im Temperaturbereich von 32.6$^\circ$C bis 33.7$^\circ$C beobachtet.
%TODO Konstantanteil, Modulationsamplitude

\subsubsection*{Stromabhängigkeit der Laserfrequenz} Der Aufbau zur Bestimmung der Stromabhängigkeit ist wie oben,
die Temperatur beträgt bei der Messung 34.0$^\circ$C.

\subsection{Auswertung}
\subsubsection*{P-I-Kennlinie der Laserdiode}

\subsubsection*{Temperaturabhängigkeit der Laserfrequenz}

\subsubsection*{Stromabhängigkeit der Laserfrequenz} 