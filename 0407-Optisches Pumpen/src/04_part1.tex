\section{Charakterisierung des Messaufbaus}
\subsection{Durchführung}
\subsubsection*{P-I-Kennlinie der Laserdiode}
Für die erste Messung werden nur die beiden Linsen im Strahlengang justiert, alle anderen Elemente sind entfernt. 
Das Peltierelement wird auf 33.9$^\circ$C eingestellt und die Spannung an der
Photodiode bei verschiedenen Laserdiodenströmen aufgezeichnet.

\subsubsection*{Temperaturabhängigkeit der Laserfrequenz}
Zur Bestimmung der Temperaturabhängigkeit der
Laserfrequenz wird das Etalon in den Strahlengang eingesetzt und der Diodenstrom mit einem Dreieckssignal moduliert
(Signalgenerator ist das im Aufbau vorhandene Netzgerät).
Der Konstantanteil des Stroms beträgt 63.6\,mA.
Modulationssignal und Photodiodensignal werden mit dem Oszilloskop gemessen und die Wanderung eines Etalonpeaks
im Temperaturbereich von 32.6$^\circ$C bis 33.7$^\circ$C beobachtet.

\subsubsection*{Zeitabhängigkeit der Laserfrequenz}
Der Aufbau zur Bestimmung der Stromabhängigkeit ist
wie bei der Messung der Temperaturabhängigkeit.
Der Konstantanteil des Laserstroms beträgt bei der Messung 64.7\,mA,
die Temperatur der Laserdiode 34.0$^\circ$C.
Die Modulation des Laserstroms findet mit dem \emph{Instec function~generator GFG-8210} statt,
da dieser eine bessere Signalqualität als das Netzgerät des Versuchsaufbaus liefert.

%TODO
Freq, Amp der Dreiecksspannung

\subsection{Auswertung}
\subsubsection*{P-I-Kennlinie der Laserdiode}
Abb Diodenkennlinie

\subsubsection*{Temperaturabhängigkeit der Laserfrequenz}
Abb T-Abhängigkeit

\subsubsection*{Zeitabhängigkeit der Laserfrequenz} 

Abb I-Abhängigkeit nur für steigende,
fallende Flanke nur Wert.

mit Rampenfit: GHz/mA (Stromabhängigkeit)