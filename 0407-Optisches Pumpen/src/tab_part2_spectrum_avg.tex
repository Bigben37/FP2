\begin{table}[H]
\caption{Theoretisches und aus den experimentellen Daten (\autoref{tab:hfs:spectrum}) gemitteltes Hyperfeinstrukturspektrum von Rubidium.}
\begin{center}
\begin{tabular}{|c|c|c|c|}
  \hline
  Übergang & $\Delta \nu^\text{theo}$ / GHz & $\Delta \nu^\text{exp}_\text{avg}$ / GHz & $s_{\Delta \nu^\text{exp}_\text{avg}}$ / GHz \\ \hline
  \rb{87}, F:2-1 & -4.81 & -4.91 & 0.08 \\ \hline
  \rb{87}, F:2-2 & -3.99 & -4.16 & 0.07 \\ \hline
  \rb{85}, F:3-2, 3-3 & -3.04 & -3.25 & 0.09 \\ \hline
  \rb{85}, F:2-2, 2-3 & 0.00 & 0.00 & 0.05 \\ \hline
  \rb{87}, F:1-1 & 2.02 & 2.07 & 0.07 \\ \hline
  \rb{87}, F:1-2 & 2.84 & 2.88 & 0.07 \\ \hline
\end{tabular}
\end{center}
\label{tab:hfs:spectrum:avg}
\end{table}
