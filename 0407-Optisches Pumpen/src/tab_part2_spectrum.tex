\begin{table}[H]
\caption{Theoretisches und (steigende und fallende Flanke) experimentell bestimmtes Hyperfeinstrukturspektrum von Rubidium.}
\begin{center}
\begin{tabular}{|c|c|c|c|c|c|}
  \hline
  Übergang & $\Delta \nu^\text{theo}$ / GHz & $\Delta \nu^\text{exp}_\text{up}$ / GHz & $s_{\Delta \nu^\text{exp}_\text{up}}$ / GHz & $\Delta \nu^\text{exp}_\text{down}$ / GHz & $s_{\Delta \nu^\text{exp}_\text{down}}$ / GHz \\ \hline
  \rb{87}, F:2$\to$1 & -4.81 & -4.86 & 0.12 & -4.95 & 0.10 \\ \hline
  \rb{87}, F:2$\to$2 & -3.99 & -4.10 & 0.09 & -4.24 & 0.10 \\ \hline
  \rb{85}, F:3$\to$2, 3$\to$3 & -3.04 & -3.24 & 0.13 & -3.25 & 0.12 \\ \hline
  \rb{85}, F:2$\to$2, 2$\to$3 & 0.00 & 0.00 & 0.07 & 0.00 & 0.07 \\ \hline
  \rb{87}, F:1$\to$1 & 2.02 & 2.15 & 0.10 & 2.01 & 0.09 \\ \hline
  \rb{87}, F:1$\to$2 & 2.84 & 2.90 & 0.09 & 2.85 & 0.10 \\ \hline
\end{tabular}
\end{center}
\label{tab:hfs:spectrum}
\end{table}
