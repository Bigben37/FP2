\section{Messung der Spin-Relaxationszeit nach Dehmelt}
\label{sect:dehmelt}
\subsection{Durchführung}
Bei dieser Methode der Bestimmung der Spin-Relaxationszeit wird die Tatsache ausgenutzt,
dass bei Beleuchtung des Ensembles der Rubidiumatome durch \textsigma$^+$-Licht
zwei konkurrierende Prozesse vorliegen, die den Orientierungsprozess der Spins und damit
die Lichttransmission durch das Ensemble beeinflussen:
Zum einen findet eine Polarisation durch den Pumpprozess statt,
zu anderen eine Depolarisation (Relaxation) durch Stöße der Gasatome mit dem Glaskolben.
Eine Variation der Stärke des Pumpprozesses durch Veränderung der Intensität des Laserlichts
erlaubt einen Rückschluss auf den konstantbleibenden Anteil der Relaxation.

Im Strahlengang befinden sich die beiden Linsen und das \textlambda/4-Plättchen.
Das vertikale Erdmagnetfeld wird mit einem Strom von 93\,mA durch Spule~4 kompensiert.
Spule~3 erzeugt ein magnetisches Wechselfeld in Richtung des Laserstrahls,
das zweimal pro Periode eine Neuorientierung des Ensembles verursacht.
Verwendet wird dazu der \emph{Instec function~generator},
die Frequenz des Rechtecksignals beträgt 50\,Hz und die Amplitude (pp) 0.1\,V.

Das Absorptionssignal der Zelle wird am Oszilloskop erst mit 100\% Beleuchtungsintensität gemessen
und anschließend mit sieben verschiedenen Intensitäten (zwischen 2\% und 40\%),
die durch unterschiedliche Abschwächungen mit ND-Filtern im Strhalengang erreicht werden.

Die Temperatur des Lasers beträgt bei der Messung 34.0$^\circ$C, der Laserstrom 65.5\,mA


\rb{85}

Laserstrom, Temp







\subsubsection*{Kalibrierung der Neutraldichtefilter}
Um die Stärke der verschiedenen Neutraldichtefilter festzustellen,
wird eine Kalibrierung mit Laser und Photodiode durchgeführt.
Die Temperatur des Lasers beträgt dabei 34.0$^\circ$C, der Laserstrom 52.0\,mA.
Die Spannung an der Photodiode wird ohne Filter gemessen und anschließend für zehn verschiedene Graufilter
mit nominellen Transmissivitäten von 0.001\% bis 50\%.
Ein weiterer Messpunkt wird mit ausgeschaltetem Laser aufgenommen.


\subsection{Auswertung}