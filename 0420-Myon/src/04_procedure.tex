\section{Experimental Procedure}


\subsection{Energy resolution: Photoelectronstatistics}
For the determination of the energy resolution of the setup, the LED in the tank is used.
\autoref{tab:conderes} shows the inputs of the coincidence units that are switched on during the measurements.
A pulse generator produces a periodic signal with a frequency of 5\,kHz.
The signal is delivered to a LED driving unit, where the intensity of the LED can be adjusted.
Twelve measurements with different intensities are conducted, in order to cover the whole energy range of the MCA\,I.
The duration of each measurement is 1\,min.
For the three lowest intensities, the measuring time is increased to 6\,min, 3\,min and 2\,min to reduce the error
on the count rate.
Each measurement produces an energy spectrum on MCA\,I which is saved.

\begin{table}[H]
\caption{Active inputs of the coincidence units during measurement of the energy resolution of the setup.}
\begin{center}
\begin{tabular}{|c|c|}
  \hline
  coincidence unit	& active inputs	\\ \hline\hline
  AND I				& A B			\\ \hline
  AND II			& -				\\ \hline
  AND III			& C				\\ \hline
 \end{tabular}
\end{center}
\label{tab:conderes}
\end{table}

\subsection{Energy calibration}

\subsubsection{Pedestal measurement}
\autoref{tab:condped} shows the state of the setup for the measurement of the pedestal.
Due to this wiring, the linear gate is opened each time a signal from PMt is delivered.
As only about every 8th event from PMt coincides with an event of PMr and PMl (\autoref{tab:countrates}),
in this configuration most likely no signal will pass the gate when it is opened.
Such events give rise to the pedestal, a large peak at the low end of the spectrum measured with MCA\,I.
After minimizing the pedestal with the offset screws on the delay generators, the pedestal is measured for
5\,min.

\begin{table}[H]
\caption{Active inputs of the coincidence units during measurement of the pedestal.}
\begin{center}
\begin{tabular}{|c|c|}
  \hline
  coincidence unit	& active inputs	\\ \hline\hline
  AND I				& -				\\ \hline
  AND II			& A				\\ \hline
  AND III			& B				\\ \hline
 \end{tabular}
\end{center}
\label{tab:condped}
\end{table}

\subsubsection{Recording of the flight through spectra}
\autoref{tab:condft} shows the settings of the coincidence units during the measurement of the flight through spectra
for the energy calibration.
In order to assign energy values to the channels of MCA\,I,
three measurements with different attenuation are conducted.
Attenuation and the duration of the measurements are given in \autoref{tab:encal}.
The measurement at 35\% signal strength is done to increase the acuracy of the energy calibration near
the expected energy of the peak of the \textbeta-spectrum.

\begin{table}[H]
\caption{Active inputs of the coincidence units during measurement of the flight through spectra.}
\begin{center}
\begin{tabular}{|c|c|}
  \hline
  coincidence unit	& active inputs	\\ \hline\hline
  AND I				& A B			\\ \hline
  AND II			& A	B C			\\ \hline
  AND III			& B	C			\\ \hline
 \end{tabular}
\end{center}
\label{tab:condft}
\end{table}
\begin{table}[H]
\caption{Attenuation and duration for the measurements of the flight through spectra.}
\begin{center}
\begin{tabular}{|c|c|c|c|}
  \hline
  measurement	& signal strength	& attenuation	& duration	\\ \hline\hline
  1				& 100\%				& -12\,dB		& 16\,h		\\ \hline
  2				& 50\%				& -18\,dB		& 16.5\,h	\\ \hline
  3				& 35\%				& -21\,dB		& 3.5\,h	\\ \hline
 \end{tabular}
\end{center}
\label{tab:encal}
\end{table}


\subsection{Time calibration}
The time calibration of the TAC and the MCA\,II is performed
with a signal generator,
which supplies two pulses with adjustable time difference on its two outputs.
The outputs are connected to the input and the output of the TAC and the signal
of the MCA is recorded.
On the oscilloscope, the time difference between input and output signal is measured.
Four measurements with time differences between 2\,\textmu s and 8\,\textmu s are done.
The duration of one measurement is 10\,min.


\subsection{Underground measurement}
\autoref{tab:condub} shows the settings of the coincidence units for the measurement of the underground
signal that occurs during the measurement of the \textbeta-spectrum and the determination of the muon lifetime.
To ensure that only statistical events are recorded, the delay of the gate/delay generator is increased by 
a factor of 100 to 50\,\textmu s. The underground is measured for 5.5\,h.

The underground signal was only recorded for the energy measurement,
since the underground for the determination of the decay time is negligible.
This is due to the very low count rate of the underground (0.016\,$\pm$\,0.004\,s$^{-1}$, \autoref{tab:countrates}).
As the TAC is active for just 10\,\textmu s after it has been started, the probability $p_\text{u}$
that during this time an underground signal in the TAC occurs is
\begin{equation}
    p_\text{u}=10\,\text{\textmu s} \cdot 0.016\,\text{s}^{-1}=1.6 \cdot 10^{-7} \ \, .
\end{equation}
The TAC is started about 21 times per second (\autoref{tab:countrates}),
so for the underground count rate $c_\text{u}$ we get
\begin{equation}
    c_\text{u}=21\,\text{s}^{-1}\cdot1.6*10^{-7}=3.4\cdot10^{-6}\,\text{s}^{-1} \ \, .
\end{equation}
This is one count every 3 days.

\begin{table}[H]
\caption{Active inputs of the coincidence units during measurement of the underground signal,
the \textbeta-spectrum and the muon lifetime.}
\begin{center}
\begin{tabular}{|c|c|}
  \hline
  coincidence unit	& active inputs	\\ \hline\hline
  AND I				& A B			\\ \hline
  AND II			& A	C			\\ \hline
  AND III			& A C			\\ \hline
 \end{tabular}
\end{center}
\label{tab:condub}
\end{table}

\subsection{The \textbeta-spectrum and the mean lifetime}

For the main measurement, the configuration of the coincidence units is shown in \autoref{tab:condub}.
The \textbeta-spectrum is collected in the MCA\,I, the decay time of the muon in MCA\,II.
The measurement runs for 65\,h.


\subsection{Schedule of the experiment}
The above description of the various parts of the experiment is arranged in a logical order.
The actual execution followed a different order which is given in \autoref{tab:exporder}.

\begin{table}[H]
\caption{Schedule of the experiment.}
\begin{center}
\begin{tabular}{|c|p{12cm}|}
  \hline
  day				&  \multicolumn{1}{|c|}{activity} 		\\ \hline\hline
  Monday			& connection of devices, examination of signals with oscilloscope			\\ \hline
  Tuesday			& testing of setup, solving a problem with a defective contact of the linear amplifier, taking pictures with oscilloscope						\\ \hline
  Wednesday			& final testing, measuring of count rates, time calibration (1\,h), flight through spectrum 100\% (over night)				\\ \hline
  Thursday			& pedestal (5\,min), underground (5.5\,h), energy resolution (1\,h), flight through spectrum 50\% (over night)		\\ \hline
  Friday			& test experiment \textbeta-spectrum and mean lifetime (1\,h), flight through spectrum 35\% (3.5\,h), \textbeta-spectrum and mean lifetime (65\,h)			\\ \hline
 \end{tabular}
\end{center}
\label{tab:exporder}
\end{table}


