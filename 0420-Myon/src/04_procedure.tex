\section{Experimental Procedure}


\subsection{Energy resolution: Photoelectronstatistics}
For the determination of the energy resolution of the setup, the LED in the tank is used.
\autoref{tab:conderes} shows the input conditions of the coincidence units during the measurements.
A pulse generator produces a periodic signal with a frequency of 5\,kHz.
The signal is delivered to a LED driving unit, where the intensity of the LED can be adjusted.
12~measurements with different intensities are conducted, in order to cover the whole energy range of the MCA.
The duration of each measurement is 1\,min.
For the three lowest intensities, the measuring time is increased to 6\,min, 3\,min and 2\,min to reduce the error
on the count rate.
Each measurement produces an energy spectrum on MCA\,I which is saved.

\begin{table}[H]
\caption{Active inputs of the coincidence units during measurement of the energy resolution of the setup.}
\begin{center}
\begin{tabular}{|c|c|}
  \hline
  coincidence unit	& active inputs	\\ \hline
  AND I				& A B			\\ \hline
  AND II			& -				\\ \hline
  AND III			& C				\\ \hline
 \end{tabular}
\end{center}
\label{tab:conderes}
\end{table}

\subsection{Energy calibration}

\subsubsection{Pedestal measurement}
\autoref{tab:condped} shows the state of the setup for the measurement of the pedestal.
Due to this wiring, the linear gate is opened each time a signal from PMt is delivered.
As only about every 8th event from PMt coincides with an event of PMr and PMl (\autoref{tab:countrates}),
in this configuration most likely no signal will pass the gate when it is opened.
Such events give rise to the pedestal, a large peak at the low end of the spectrum measured with MCA\,I.
After minimizing the pedestal with the offset screws on the delay generators, the pedestal is measured for
5\,min.

\begin{table}[H]
\caption{Active inputs of the coincidence units during measurement of the pedestal.}
\begin{center}
\begin{tabular}{|c|c|}
  \hline
  coincidence unit	& active inputs	\\ \hline
  AND I				& -				\\ \hline
  AND II			& A				\\ \hline
  AND III			& B				\\ \hline
 \end{tabular}
\end{center}
\label{tab:condped}
\end{table}

\subsubsection{Flight through spectrum}


\subsection{Time calibration}


\subsection{Underground measurement}

\subsection{\textbeta-spectrum and mean lifetime}


energycal
pedestal,flight through

timecal

underground

betaspectrum+lifetime
