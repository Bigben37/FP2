\section{Physical principles}
\subsection{Standard Model}
The \emph{Standard Model of particle physics} describes all known elementary particles (\autoref{img:standardmodel}) and the different 
interactions between them.
\begin{figure}[H]
        \centering
        \def\svgwidth{0.55\textwidth}
       \input{../img/standardmodel.pdf_tex}
        \caption{The known particles of the Standard Model.}
        \label{img:standardmodel}
\end{figure}
The four known interactions are gravitation, electromagnetism, strong and weak interaction. Not all elementary particles obey all interactions.\\
Elementary particles are the smallest particles currently known, until now there was no substructure found. 
This doesn't mean that they are stable or free observable. They divide into fermions (spin $\frac{1}{2}$) that make up matter and 
gauge bosons (spin 1), listed in \autoref{tab:gaugebosons}, which carry the fundamental interactions. 
\begin{table}[H]
\caption{Gauge bosons and their interaction.}
\begin{center}
\begin{tabular}{|c|c|}
	\hline
	gauge boson 		& interaction			\\ \hline \hline
	photon (\textgamma) & electromagnetic 		\\ \hline
	gluon (g) 			& strong				\\ \hline
	Z-boson ($Z^0$)		& weak					\\ \hline
	W-boson ($W^\pm$)	& weak					\\ \hline
\end{tabular}
\end{center}
\label{tab:gaugebosons}
\end{table}
Furthermore there is the Higgs boson, which is not a direct result of a gauge theory and 
therefore doesn't carry a fundamental force. But it is needed to combine the electromagnetic and weak interactions and to give the Z- and W-bosons
their mass. \\
Fermions can be divided into six quarks, which interact via the strong interaction, and six leptons, which don't. Both quarks and fermions are 
divided into three generations, each with two particles. Equivalent particles of those generations have almost the same properties except for 
their mass.
\subsection{The muon}
Muons are charged ($z=\pm1$) leptons of the second generation. They have a mass\footnote{\url{http://physics.nist.gov/cgi-bin/cuu/Value?mmuc2mev|search_for=muon} (visited May 9, 2015)} of 
\begin{equation}
	\label{eq:litval:mass}
    E_\mu = (105.6583715 \pm 0.0000035)\,\text{MeV}
\end{equation}
and a mean lifetime\footnote{J. Beringer \emph{et al.} (Particle Data Group), PR \textbf{D86}, 010001 (2012) and 2013 partial update for the 2014 edition} of 
\begin{equation}
	\label{eq:litval:meanlifetime}
    \tau_\mu = \left( 2.1969811 \pm 0.0000022 \right)\,\text{\textmu s}\ \, .
\end{equation}
\subsubsection{Muon decay}
\label{sub:decay}
Muons decay into electrons or positrons (\autoref{img:feynman_mu-to-e}):
\begin{equation}
	\mu^- \rightarrow e^- + \nu_\mu + \bar{\nu}_e \qquad \qquad \mu^+ \rightarrow e^+ + \bar{\nu}_\mu + \nu_e
\end{equation}
\begin{figure}[H]
        \centering
        \def\svgwidth{0.35\textwidth}
       \input{../img/feynman.pdf_tex}
        \caption{$\mu^-$ decay into $e^-$, $\nu_\mu$ and $\bar{\nu}_e$.}
        \label{img:feynman_mu-to-e}
\end{figure}
This is a three body decay, the energy spectrum corresponds to normal \textbeta-decay spectrum. \\
The time dependency of a decaying sample of muons can be described with the normal exponential decay law:
\begin{equation}
	\label{eq:expdecay}
	\frac{\difd N}{\difd t} = - \lambda N \qquad \Rightarrow \qquad N(t) \propto e^{-\lambda t}
\end{equation}
where $N$ is the number of particles and $\lambda$ the exponential decay constant. The mean lifetime calculates to:
\begin{equation}
    \tau := \frac{\left<t\right>}{\int_0^\infty N(t) \difd t} = 
    \frac{\int_0^\infty t e^{-\lambda t} \difd t}{\int_0^\infty e^{-\lambda t} \difd t} = 
    \frac{\frac{1}{\lambda^2}}{\frac{1}{\lambda}} = \frac{1}{\lambda}
\end{equation}
After this time only $\frac{1}{e} \approx 37\%$ of an inital amount of muons exist. The mean lifetime applies to the rest frame of the muon.
\subsubsection{Origin of muons}
The information in this section is based on \cite{kosmische-strahlung} and \cite{staatsex}.\\
Muons are produced by cosmic rays in the earth's atmosphere. There are two different kinds of radiations, the primary radiation and the secondary radiation.
\paragraph{Primary radiation}
The origin of primary radiation is not clear, but there are assumptions that the particles originate from supernovae and pulsars.
Protons make the largest component of primary radiation with 92\%. Furthermore it consists of helium cores (\textalpha-particles) with a percentage 
of 7\% and electrons with 1\%. Heavy cores are rarely measured. The energy spectrum ranges from $10^7$\,eV to $10^{20}$\,eV, whereas particles with 
less energy are more common.
\paragraph{Secondary radiation}
While entering the atmosphere primary particles collide with the atoms of the atmosphere. A variety of generation and annihilation processes occur. 
This happens usually in a height of 15-20\,km.
Important for this experiment is the generation of pions and kaons:
\begin{equation}
    \text{p} + \text{nucleus} \rightarrow \pi^{0, \pm}, K^{0, \pm}, \text{n}, \text{p}
\end{equation}
The charged pions and kaons decay after a short time, some into muons:
\begin{equation}
    \begin{split}
        & \pi^+ \rightarrow \mu^+ + \nu_\mu \\
        & \pi^- \rightarrow \mu^- + \bar{\nu}_\mu \\
        & K^+ \rightarrow \mu^+ + \nu_\mu \\
        & K^- \rightarrow \mu^- + \bar{\nu}_\mu
    \end{split}
\end{equation}
There are more positive muons than negative ones, this is caused by two reasons: It is far more probable that a proton, which has 
a positive charge, generates also a positive particle by interaction with the atoms of the atmosphere. Furthermore, like electron capture, negative muons can be 
caught by a nucleus.
\subsubsection{Relativistic muons}
With classical physics one would not expect the muons to reach the ground. With a mean lifetime of approximately $\tau=2.2$\,\textmu s they would 
travel with a velocity close to light speed only a distance of  $\tau c = 660$\,m. The reason why muons can be detected anyhow is special relativity. Muons generated 
by cosmic rays have such a high energy that their mean lifetime observed from earth increases by a 
factor\footnote{The Lorentz factor. Here $\gamma$ can be calculated with $\gamma = \frac{E}{E_0}$, where $E$ is the energy and $E_0$ the rest energy of the muon.}  
of $\gamma$
because of time dilation. For example: A muon with $E=10.5$\,GeV has a observed mean lifetime of $\tau'=\gamma \tau=218.6$\,\textmu s and travels 65.6\,km on average. \\
Now one can also calculate the percentage $\frac{N}{N_0}$ of muons ($E=10.5$\,GeV) which reach the ground with \autoref{eq:expdecay}:
\begin{equation}
    \frac{N}{N_0} = e^{-\frac{t}{\tau'}} = e^{-\frac{s}{v\tau'}}
\end{equation}
With $s=15$\,km and $v \approx c$ the calculation yields $\frac{N}{N_0} = 79.6\%$.
\subsection{Energy loss via ionization and excitation}
When particles interact with matter they can lose energy via elastic collisions or through excitation or ionization of the electrons in the 
atomic shell. This effect is used to detect particles, for example it is utilized by the scintillators in this experiment. The average energy loss per path length 
$-\frac{\difd E}{\difd x}$ can be calculated using the \emph{Bethe formula}. The following formula applies to relativistic particles (\cite{dem4}, p. 87-89, formula slightly remodeled):
\begin{equation}
    - \frac{\difd E}{\difd x} = \frac{4 \pi}{m_e c^2} \cdot \frac{n_e z^2}{\beta^2} \cdot \left( \frac{e^2}{4 \pi \epsilon_0} \right)^2 \cdot \left[ \ln \left( \frac{2 m_e c^2 \beta^2}{E_b \left( 1-\beta^2 \right)  } \right) - \beta^2 \right]
\end{equation}
Nomenclature: 
\begin{itemize}
  \item detector: electron density $n_e$, mean binding energy of electrons $E_b$
  \item particle: charge number $z$, ratio of velocity to the speed of light $\beta = \frac{v}{c}$
  \item constants: mass of electron $m_e$, vacuum permittivity $\epsilon_0$, speed of light $c$
\end{itemize}
Two notable dependencies of $-\frac{\difd E}{\difd x}$ are the quadratic one on the charge, $-\frac{\difd E}{\difd x} \sim z^2$ and the inverse 
quadratic one on the relative velocity, $-\frac{\difd E}{\difd x} \sim \frac{1}{\beta^2}$. There is no dependency on the mass of the incidental 
particle.\\
The qualitative graph of the \emph{Bethe formula} is shown in \autoref{img:bethe}. 
\begin{figure}[H]
        \centering
        \def\svgwidth{0.55\textwidth} 
       \input{../img/bethebloch.pdf_tex}
        \caption{Bethe formula: Average energy loss per path length $-\frac{\difd E}{\difd x}$ of a particle with relative velocity $\beta$.}
        \label{img:bethe}
\end{figure}
For small values of $\beta$ the graph behaves like $\sim \frac{1}{\beta^2}$, for large values like 
$\sim \ln \left( \frac{\beta^2}{1-\beta^2} \right)$. In between there is a minimum, those particles with the minimal energy loss per length are called 
\emph{minimal ionizing particles}.
\subsection{(V-A)-theory}
The (V-A)-theory\footnote{vector-axialvector} is a model for the weak interaction based on Fermi's theory of weak interaction. Today it is 
used as a low energy approximation for the latest theory. Since the theory is quite complicated and is not in the curriculum of a bachelor's 
degree a reference for the calculation of the muon decay is made to \cite{staatsex} (p. 19-23). The formula for the mean lifetime $\tau_\mu$ of a muon is
\begin{equation}
    \tau_\mu = \frac{192 \pi^3 \hbar^7}{G_\mu^2 m_\mu^5 c^4}\ \, .
\end{equation}
If the mass $m_\mu$ and the mean lifetime $\tau_\mu$ of the muon is known, the weak coupling constant $G_\mu$ can be calculated:
\begin{equation}
    \label{eq:weakcouplingconstant}
    G_\mu = \sqrt{\frac{192 \pi^3 \hbar^7}{\tau_\mu m_\mu^5 c^4}}
\end{equation}
The literature value\footnote{\url{http://physics.nist.gov/cgi-bin/cuu/Value?gf} (visited May 9, 2015)} is:
\begin{equation}
    \label{eq:litval:weakcouplingconstant}
    G_\mu = \left( 1.166364 \pm 0.000005 \right) \cdot 10^{-5}\,\frac{1}{\text{GeV}^2}
\end{equation}


\subsection{Scintillators}
Muons can be detected using scintillating materials.
Here a short introduction to the mechanism of scintillation will be given
(the information is according to \cite{dem4} and \cite{staatsex}).

In scintillators the interaction of high-energy particles and matter is used to determine
the energy of the particle:
As it passes through the scintillating material, it excites electrons.
The excitation energy amounts to several electronvolts and the number of excited electrons is (in the ideal case
of complete absorption) proportional to the energy of the particle.
To be able to measure the ultraviolet light that is emitted when the electrons fall back into their ground state,
the scintillator has to be transparent for light of this wavelength.
This is achieved using a wavelength shifter, a substance that exhibits a large \emph{stokes shift}:
If an electron in the ground state is excited, it enters a higher energy level.
Additionally, vibrational states of the molecule are excited.
A small amount of energy is emitted promptly through vibrational relaxation,
so the energy of the photon that is produced when the electron falls back in the ground state
is smaller than the absorbed photon (stokes shift).
For the emitted photon, the scintillator is transparent,
thus it can exit and be detected by a photomultiplier.





\subsection{Photomultipliers}

\begin{figure}[H]
        \centering
        \def\svgwidth{0.6\textwidth}
       \input{../img/photomultiplier.pdf_tex}
        \caption{Composition of a photomultiplier for the transduction of weak light signals into current pulses.}
        \label{img:pm}
\end{figure}

The light signal delivered by a scintillating material is very faint, thus it needs intensification.
Photomultipliers are used for this task.

In a photomultiplier, photons hit a photocathode and create free electrons by reason of the photoeffect.
The electrons are accelerated in an electric field to the first dynode of the photomultiplier (\autoref{img:pm}).
The following dynodes are connected to a voltage divider in a way that the next dynode has a more positive potential.
Thereby the avalanche of electrons that is provoked by a few photons is amplified:
An electron that hits a dynode generates several free electrons.
This effect produces an output signal at the end of the photomultiplier whose intensity is proportional
to the number of incoming photons.
