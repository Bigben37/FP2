\section{Physical Principles}
\subsection{Standard Model}
The \emph{Standard Model of Particle Physics} describes all known elementary particles (\autoref{img:standardmodel}) and the different 
interactions between them. \\
%TODO: Bild Standardmodell http://de.wikipedia.org/wiki/Datei:Standard_Model_of_Elementary_Particles-de.svg übernehmen oder selber malen)
The four known interactions are gravitation, electromagnetism, strong and weak interaction. Not all elementary particles obey all interactions.\\
Elementary particles are the smallest particles currently known, until now there was no substructure found. 
This doesn't mean that they are stable or free observable. They divide into fermions (spin $\frac{1}{2}$) that make up matter and 
gauge bosons (spin 1), listed in \autoref{tab:gaugebosons}, which carry the fundamental interactions. 
\begin{table}[H]
\caption{Gauge bosons and their interaction.}
\begin{center}
\begin{tabular}{|c|c|}
	\hline
	gauge boson 		& interaction			\\ \hline \hline
	photon (\textgamma) & electromagnetic 		\\ \hline
	gluon (g) 			& strong				\\ \hline
	Z-boson ($Z^0$)		& weak					\\ \hline
	W-boson ($W^\pm$)	& weak					\\ \hline
\end{tabular}
\end{center}
\label{tab:gaugebosons}
\end{table}
Furthermore there is the Higgs boson, which is not a direct result of a gauge theory and 
therefore doesn't carry a fundamental force. But it is needed to combine the electromagnetic and weak interaction and to give the Z- and W-bosons
a mass. \\
Fermions can be divided into six quarks, which interact via the strong interaction, and six leptons, which don't. Both quarks and fermions are 
divided into three generations of respectively two particles. Equivalent particles of those generations have almost the same properties except for 
their mass.
\subsection{Muon}
Muons are charged ($z=\pm1$) leptons of the second generation. They have a mass\footnote{\url{http://physics.nist.gov/cgi-bin/cuu/Value?mmuc2mev|search_for=muon} (visited May 9. 2015)} of 
\begin{equation}
	\label{eq:litval:mass}
    E_\mu = (105.6583715 \pm 0.0000035)\,\text{MeV}
\end{equation}
and a mean lifetime\footnote{J. Beringer \emph{et al.} (Particle Data Group), PR \textbf{D86}, 010001 (2012) and 2013 partial update for the 2014 edition} of 
\begin{equation}
	\label{eq:litval:meanlifetime}
    \tau_\mu = \left( 2.1969811 \pm 0.0000022 \right)\,\text{\textmu s}
\end{equation}
\subsubsection{Muon decay}
\label{sub:decay}
Muons decay into electrons /positrons (\autoref{img:feynman_mu-to-e})
\begin{equation}
	\mu^- \rightarrow e^- + \nu_\mu + \bar{\nu}_e \qquad \qquad \mu^+ \rightarrow e^+ + \bar{\nu}_\mu + \nu_e
\end{equation}
TODO Feynman \\%TODO Feynman mu -> e + nu_e + nu_mu
This is a three body decay, the energy spectrum corresponds to normal \textbeta-decay spectrum. \\
The time dependency of a decaying sample of muons can be described with the normal exponential decay law:
\begin{equation}
	\label{eq:expdecay}
	\frac{\difd N}{\difd t} = - \lambda N \qquad \Rightarrow \qquad N(t) \propto e^{-\lambda t}
\end{equation}
where $N$ is the number of particles and $\lambda$ the exponential decay constant. The mean lifetime calculates to:
\begin{equation}
    \tau := \frac{\left<t\right>}{\int_0^\infty N(t) \difd t} = 
    \frac{\int_0^\infty t e^{-\lambda t} \difd t}{\int_0^\infty e^{-\lambda t} \difd t} = 
    \frac{\frac{1}{\lambda^2}}{\frac{1}{\lambda}} = \frac{1}{\lambda}
\end{equation}
After this time only $\frac{1}{e} \approx 37\%$ of a inital count of muons exist. The mean lifetime applies to the rest frame of the muon.
\subsubsection{Origin of muons}
The information in this section is based on \cite{kosmische-strahlung} and \cite{staatsex}.\\
Muons are caused by cosmic rays in the earth's atmosphere. There are two different radiations, the primary radiation and the secondary radiation.
\paragraph{Primary radiation}
The origin of primary radiation is not clear, but there are assumptions that the particles originate from supernovae and pulsars.
Protons make the highest proportion of primary radiation with 92\%. Furthermore it consists of helium cores (\textalpha-particles) with a percentage 
of 7\% and electrons with 1\%. Heavy cores are rarely measured. The energy spectrum ranges from $10^7$\,eV to $10^{20}$\,eV, whereas particles with 
less energy are more probable.
\paragraph{Secondary radiation}
While entering the atmosphere primary particles collide with the atoms of the atmosphere. A variety of generation and annihilation processes occur. 
This happens usually in a height of 15-20\,km.
Important for this experiment is the generation of pions and kaons:
\begin{equation}
    p + \text{core} \rightarrow \pi^{0, \pm}, K^{0, \pm}, n, p
\end{equation}
The charged pions and kaons decay after a short time, some into muons:
\begin{equation}
    \begin{split}
        & \pi^+ \rightarrow \mu^+ + \nu_\mu \\
        & \pi^- \rightarrow \mu-+ + \bar{\nu}_\mu \\
        & K^+ \rightarrow \mu^+ + \nu_\mu \\
        & K^- \rightarrow \mu^- + \bar{\nu}_\mu
    \end{split}
\end{equation}
There are more positive muons than negative ones, this is caused by two reasons: It is far more probable that a proton, which of course has 
a positive charge, generates also a positive particle by interaction with the atoms of the atmosphere. Also TODO 2nd reason. %TODO 2. Grund
\subsubsection{Relativistic muons}
With classical physics one would not expect the muons to reach the ground. With a mean lifetime of around $\tau=2.2$\,\textmu s they would 
travel with a velocity close to light speed only a distance of  $\tau c = 660$\,m. The reason why muons can be detected anyhow is special relativity. Muons generated 
by cosmic rays have such a high energy that their mean lifetime observed from earth increases by a factor of 
$\gamma$\footnote{The Lorentz factor. Here $\gamma$ can be calculated with $\gamma = \frac{E}{E_0}$, where $E$ is the energy and $E_0$ the rest energy of the muon.} 
because of time dilation. For example: A muon with $E=10.5$\,GeV has a observed mean lifetime of $\tau'=\gamma \tau=218.6$\,\textmu s and travels 65.6\,km. \\
Now one can also calculate the percentage $\frac{N}{N_0}$ of muons ($E=10.5$\,GeV) which reach the ground with \autoref{eq:expdecay}:
\begin{equation}
    \frac{N}{N_0} = e^{-\frac{t}{\tau'}} = e^{-\frac{s}{v\tau'}}
\end{equation}
With $s=15$\,km and $v \approx c$ the calculation yields $\frac{N}{N_0} = 79.6\%$.
\subsection{Energy loss by ionization and excitation}
When particles interact with matter they can lose energy caused by elastic collisions or by excitation or ionization of the electrons in the 
atomic shell. This effect is used to detect particles, for example it is utilized by the scintillators in this experiment. The average energy loss per path length 
$-\frac{\difd E}{\difd x}$ can be calculated using the \emph{Bethe formula}. The following formula applies to relativistic particles (\cite{dem4}, p.87-89, formula slightly remodeled):
\begin{equation}
    - \frac{\difd E}{\difd x} = \frac{4 \pi}{m_e c^2} \cdot \frac{n_e z^2}{\beta^2} \cdot \left( \frac{e^2}{4 \pi \epsilon_0} \right)^2 \cdot \left[ \ln \left( \frac{2 m_e c^2 \beta^2}{E_b \left( 1-\beta^2 \right)  } \right) - \beta^2 \right]
\end{equation}
Nomenclature: 
\begin{itemize}
  \item detector: electron density $n_e$, mean binding energy of electrons $E_b$
  \item particle: charge number $z$, ratio of velocity to the speed of light $\beta = \frac{v}{c}$
  \item constants: mass of electron $m_e$, vacuum permittivity $\epsilon_0$, speed of light $c$
\end{itemize}
Two notable dependencies of $-\frac{\difd E}{\difd x}$ are the quadratic one of the charge, $-\frac{\difd E}{\difd x} \sim z^2$ and the inverse 
quadratic one of the relative velocity, $-\frac{\difd E}{\difd x} \sim \frac{1}{\beta^2}$. There is no dependency of the mass of the incidental 
particle.\\
The qualitative graph of the \emph{Bethe formula} is shown in \autoref{img:bethe}. 
%TODO Bild Bethe.
For small values of $\beta$ the graph behaves like $\sim \frac{1}{\beta^2}$, for large values like 
$\sim \ln \left( \frac{\beta^2}{1-\beta^2} \right)$. In between there is a minimum, those particles with the minimal energy loss per length are called 
\emph{minimal ionizing particles}.
\subsection{(V-A)-Theory}
The (V-A)-theory\footnote{Vector-axialvector} is a model for the weak interaction based on Fermi's theory of weak interaction. Today it is 
used as a low energy approximation for the latest theory. Since the theory is quite complicated and is not in the curriculum of a bachelor's 
degree a reference for the calculation of the muon decay is made to \cite{staatsex} (p. 19-23). The formula for the mean lifetime $\tau_\mu$ of a muon is
\begin{equation}
    \tau_\mu = \frac{192 \pi^3 \hbar^7}{G_\mu^2 m_\mu^5 c^4}
\end{equation}
If the mass $m_\mu$ and the mean lifetime $\tau_\mu$ of the muon is known, the weak coupling constant $G_\mu$ can be calculated:
\begin{equation}
    \label{eq:weakcouplingconstant}
    G_\mu = \sqrt{\frac{192 \pi^3 \hbar^7}{\tau_\mu m_\mu^5 c^4}}
\end{equation}
The literature value\footnote{\url{http://physics.nist.gov/cgi-bin/cuu/Value?gf} (visited May 9. 2015)} is:
\begin{equation}
    \label{eq:litval:weakcouplingconstant}
    G_\mu = \left( 1.166364 \pm 0.000005 \right) \cdot 10^{-5}\,\frac{1}{\text{GeV}^2}
\end{equation}


\subsection{Scintillators}


\subsection{Photomultipliers}

\begin{figure}[H]
        \centering
        \def\svgwidth{0.7\textwidth}
       \input{../img/photomultiplier.pdf_tex}
        \caption{Composition of a photomultiplier for the transduction of weak light signals into current pulses.}
        \label{img:pm}
\end{figure}

In a photomultiplier, photons hit a photocathode and create free electrons by reason of the photoeffect.
Those electrons are accelerated in an electric field to the first dynode of the photomultiplier (\autoref{img:pm}).
The following dynodes are connected to a voltage divider in a way that the next dynode has a more positive potential.
Thereby the avalanche of electrons that is provoked by a photon is amplified:
An electron that hits a dynode generates several free electrons.
This effect produces an output signal at the end of the photomultiplier whose intensity is proportional
to the number of incoming photons.
