\section{Physical Principles}
\subsection{Standard Model}
The Standard Model of Particle Physics describes all known elementary particles and the different interactions between them. 
(TODO It is one of the most/best tested (current) theories). Elementary particle are the smallest particles currently known, until now 
there can be no substructure found. This doesn't mean that they are stable (\textmu, \texttau, \ldots) or free observable (quarks). They divide into fermions that make up matter and 
gauge bosons, which carry the fundamental interactions.\\
lepton-universality\\
TODO: Bild Standardmodell (\url{http://de.wikipedia.org/wiki/Datei:Standard_Model_of_Elementary_Particles-de.svg} übernehmen oder selber malen)
\subsection{Muon}
Muons are charged leptons of the second family. They have a mass of 
$E_\mu = (105.6583715 \pm 0.0000035)\,\text{MeV}$\footnote{nist} 
and a mean lifetime of 
$\tau_\mu = \left( 2.1969811 \pm 0.0000022 \right)\,\text{\textmu s}$\footnote{J. Beringer \emph{et al.} (Particle Data Group), PR \textbf{D86}, 010001 (2012) and 2013 partial update for the 2014 edition}, 
since they are not stable.
\subsubsection{Muon decay}
\label{sub:decay}
Muons decay into electrons /positrons (\autoref{img:feynman_mu-to-e})
\begin{equation}
	\mu^- \rightarrow e^- + \nu_\mu + \bar{\nu}_e \qquad \qquad \mu^+ \rightarrow e^+ + \bar{\nu}_\mu + \nu_e
\end{equation}
TODO Feynman \\%TODO Feynman mu -> e + nu_e + nu_mu
This is a \textbeta-decay, a three-body-decay. The energy spectrum corresponds to normal \textbeta-decay spectrum. \\
The time dependency of a decaying sample of muons can be described with the normal exponential decay law:
\begin{equation}
	\label{eq:expdecay}
	\frac{\difd N}{\difd t} = - \lambda N \qquad \Rightarrow \qquad N(t) \propto e^{-\lambda t}
\end{equation}
where $N$ is the number of particles and $\lambda$ the exponential decay constant. The mean lifetime calculates to:
\begin{equation}
    \tau := \frac{\left<t\right>}{\int_0^\infty N(t) \difd t} = 
    \frac{\int_0^\infty t e^{-\lambda t} \difd t}{\int_0^\infty e^{-\lambda t} \difd t} = 
    \frac{\frac{1}{\lambda^2}}{\frac{1}{\lambda}} = \frac{1}{\lambda}
\end{equation}
After this time only $\frac{1}{e} \approx 37\%$ of a inital count of muons exist. the mean lifetime applies to the rest frame of the muon.
\subsubsection{Origin of muons}
The information in this section is based on \cite{kosmische-strahlung} and \cite{staatsex}.\\
Muons are caused by cosmic rays in the earth's atmosphere. There are two different radiations, the primary radiation and the secondary radiation.
\paragraph{Primary radiation}
The origin of primary radiation is not clear, but there are assumptions that the particles originate from supernovae and pulsars.
Protons make the highest proportion of primary radiation with 92\%. Furthermore it consists of helium cores (\textalpha-particles) with a percentage 
of 7\% and electrons with 1\%. Heavy cores are rarely measured. The energy spectrum ranges from $10^7$\,eV to $10^{20}$\,eV, whereas particles with 
less energy are more probable.
\paragraph{Secondary radiation}
While entering the atmosphere primary particles collide with the atoms of the atmosphere. A variety of generation and annihilation processes occur. 
This happens usually in a height of 15-20\,km.
Important for this experiment is the generation of pions and kaons:
\begin{equation}
    p + \text{core} \rightarrow \pi^{0, \pm}, K^{0, \pm}, n, p
\end{equation}
The charged pions and kaons decay after a short time, some into muons:
\begin{equation}
    \begin{split}
        & \pi^+ \rightarrow \mu^+ + \nu_\mu \\
        & \pi^- \rightarrow \mu-+ + \bar{\nu}_\mu \\
        & K^+ \rightarrow \mu^+ + \nu_\mu \\
        & K^- \rightarrow \mu^- + \bar{\nu}_\mu
    \end{split}
\end{equation}
There are more positive muons than negative ones, this is caused by two reasons: It is far more probable that a proton, which of course has 
a positive charge, generates also a positive particle by interaction with the atoms of the atmosphere. Also TODO 2nd reason. %TODO 2. Grund
\subsubsection{Relativistic muons}
With classical physics one would not expect the muons to reach the ground. With a mean lifetime of around $\tau=2.2$\,\textmu s they would 
travel with light speed only a distance of  $\tau c = 660$\,m. The reason why muons can be detected anyhow is special relativity. Muons generated 
by cosmic rays have such a high energy that their mean lifetime observed from earth increases by a factor of 
$\gamma$\footnote{The Lorentz factor. Here $\gamma$ can be calculated with $\gamma = \frac{E}{E_0}$, where $E$ is the energy and $E_0$ the rest energy of the muon.} 
because of time dilation. For example: A muon with $E=10.5$\,GeV has a observed mean lifetime of $\tau'=\gamma \tau=218.6$\,\textmu s and travels 65.6\,km. \\
Now one can also calculate the percentage $\frac{N}{N_0}$ of muons ($E=10.5$\,GeV) which reach the ground with \autoref{eq:expdecay}:
\begin{equation}
    \frac{N}{N_0} = e^{-\frac{t}{\tau'}} = e^{-\frac{s}{v\tau'}}
\end{equation}
With $s=15$\,km and $v \approx c$ the calculation yields $\frac{N}{N_0} = 79.6\%$.
\subsection{Energy loss by ionization and excitation}
When particles interact with matter they can lose energy caused by elastic collisions or by excitation or ionization of the electrons in the 
atomic shell. This effect is used to detect particles, it is utilized by the scintillators in this experiment. The average energy loss per path length 
$\frac{\difd E}{\difd x}$ can be calculated using the \emph{Bethe formula}. The following formula applies to relativistic particles (\cite{dem4}, p.87-89, formula slightly remodeled):
\begin{equation}
    \frac{\difd E}{\difd x} = - \frac{4 \pi}{m_e c^2} \cdot \frac{n_e z^2}{\beta^2} \cdot \left( \frac{e^2}{4 \pi \epsilon_0} \right)^2 \cdot \left[ \ln \left( \frac{2 m_e c^2 \beta^2}{E_b \left( 1-\beta^2 \right)  } \right) - \beta^2 \right]
\end{equation}
Nomenclature: 
\begin{itemize}
  \item detector: electron density $n_e$, mean binding energy of electrons $E_b$
  \item particle: charge number $z$, ratio of velocity to the speed of light $\beta = \frac{v}{c}$
  \item constants: mass of electron $m_e$, vacuum permittivity $\epsilon_0$, speed of light $c$
\end{itemize}
\subsection{V-A-Theory}
weak coupling constant