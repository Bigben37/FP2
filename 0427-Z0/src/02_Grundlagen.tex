\section{Physikalische Grundlagen}
\subsection{Standardmodell}
\subsubsection*{Teilchen}

\begin{figure}[H]
        \centering
        \def\svgwidth{0.55\textwidth}
       \input{../img/standardmodel.pdf_tex}
        \caption{Die Teilchen im Standardmodell.}
        \label{img:standardmodel}
\end{figure}

\subsubsection*{Kräfte}
\paragraph{Gravitation}
\paragraph{Starke Wechselwirkung}
\paragraph{Elektroschwache Wechselwirkung}
\subsection{Wirkungsquerschnitt}
Klassisch gesehen beschreibt der Wirkungsquerschnitt $\sigma$ bei einer Kollision von zwei Objekten die Fläche, an der sie sich treffen. 
In der Teilchenphysik gibt der Wirkungsquerschnitt eine Wahrscheinlichkeit, ob ein gewisser Prozess bei der Kollision von Teilchen auftritt, an.
Die Einheit des Wirkungsquerschnitts ist \emph{barn} ($1\,\text{b} = 10^{-24}\,\text{cm}^2$). \\
Jedoch wird im Experiment nicht direkt der Wirkungsquerschnitt gemessen, sondern die Anzahl eines gewissen Ereignisses (Zerfallskanal). Mit Hilfe 
der \emph{Luminosität} $L$, ein Maß für die Anzahl der Kollisionen pro Fläche und Zeit, kann der Wirkungsquerschnitt bestimmt werden:
\begin{equation}
    \sigma = \frac{N}{\int L \, \difd t}
\end{equation}
Dabei ist $\int L \, \difd t$ die (über die Zeit) integrierte Luminosität. \\
Wird der Wirkungsquerschnitt eines Zerfallskanals gegen die Schwerpunktsenergie $\sqrt{s}$ der kollidierenden Teilchen aufgetragen, so ist  
bei manchen Energien eine deutliche Erhöhung des Wirkungsquerschnitts zu erkennen. Dieser Effekt wird Resonanz genannt und deutet auf ein Teilchen 
hin, welches die Energie besitzt, bei der die Erhöhung des Wirkungsquerschnitts stattfindet. 
So gibt es zum Beispiel bei ungefähr 91\,GeV einen Resonanzpeak, der dem \Z-Boson zugeordnet wird. \\
Der Verlauf der  \Z-Resonanzkurve eines fermionischen Zerfallskanals kann mit einer relativistischen Breit-Wigner Kurve (analog zur Resonanz des 
klassischen harmonischen Oszillators) beschrieben werden \cite{manual}:
\begin{equation}
    \label{eq:sigma:fermion}
    \sigma_f(s) = \frac{12 \pi}{M_\text{Z}^2} \cdot \frac{s \Gamma_\text{e} \Gamma_f}{ \left( s - M_\text{Z} \right)^2 + \left( s^2 \Gamma_\text{Z}^2 / M_\text{Z}^2 \right) }
\end{equation}
$M_\text{Z}$ ist die Masse des \Z-Bosons, $\Gamma_\text{e}$ die elektronische Zerfallsbreite und $\Gamma_f$ die Zerfallsbreite des Fermions. Auf 
die Zerfallsbreiten wird genauer in \autoref{sub:theo:gamma} eingegangen.
\subsection{\texorpdfstring{\elp-\elm}{e+/e-}-Wechselwirkung}
\subsubsection*{Zerfallskanäle}
Im LEP (siehe \autoref{sec:setup}) werden Kollisionen von Elektronen und Positronen untersucht.
Vernichten sich \elp und \elm\ (s-Kanal\footnote{Die Bezeichnungen s-Kanal (Annihilation) und t-Kanal (Streuung) gehen auf die 
\emph{Mandelstammvariablen} zurück.}), so kann entweder ein Photon oder ein \Z-Boson entstehen, welche beide nach kurzer Zeit wieder 
zerfallen. Bei einer Schwerpunktsenergie $\sqrt{s}$ in der Nähe der \Z-Masse ist die Produktion von Photonen jedoch stark unterdrückt, 
weshalb diese Zerfallskanäle nicht relevant für die Bestimmung von Eigenschaften des \Z-Bosons sind. \\
Das \Z-Boson kann in $\ell^+\ell^-$-, $\nu_\ell\bar{\nu}_\ell$- und $q\bar{q}$-Paare zerfallen (\autoref{img:Z0:decay}).
Dabei bezeichnet $\ell$ eines der drei Leptonen (e, \textmu, \texttau) und $q$ eines der fünf leichtesten Quarks (u, d, c, s, b). Ein \xx{t}-Paar 
kann nicht erzeugt werden, da die Masse des top-Quarks größer ist als die von \Z.
\begin{figure}[H]
        \centering
        \def\svgwidth{0.7\textwidth}
       \input{../img/Z0decay.pdf_tex}
        \caption{Zerfallskanäle des \Z-Bosons.}
        \label{img:Z0:decay}
\end{figure}
\subsubsection*{Bhabha-Streuung}
\label{subsub:theo:stchannel}
Neben der Annihilation von \ee gibt es auch die \ee-Streuung
(der t-Kanal der \emph{Bha\-bha-Streu\-ung}, \autoref{img:bhabha}).
Die Bhabha-Streuung wird durch die Reaktion \ee$\to$\ee\ beschrieben.
Da für die experimentelle Bestimmung der \Z-Breite nur der s-Kanal relevant ist, müssen die Ereignisse des t-Kanals gefiltert werden. 
Dies ist durch eine Unterscheidung der Zerfallsprodukte nicht möglich, da sie in beiden Kanälen gleich sind. Allerdings besitzten 
der Wirkungsquerschnitte von s- und t-Kanal eine unterschiedliche Winkelabhängigkeit des Polarwinkels $\Theta$ 
(die Strahlachse entspricht der $z$-Achse). Es gilt~\cite{manual}:
\begin{equation}
    \label{eq:stchannel:sigmas}
    \frac{\difd \sigma_\text{s}}{\difd \Omega} \sim \left( 1 + \cos^2 \Theta \right), \qquad 
    \frac{\difd \sigma_\text{t}}{\difd \Omega} \sim \left( 1 - \cos \Theta \right)^{-2}
\end{equation}
\begin{figure}[H]
        \centering
        \def\svgwidth{0.5\textwidth}
       \input{../img/tchanBhabha.pdf_tex}
        \caption{Der t-Kanal der Bhabha-Streuung.}
        \label{img:bhabha}
\end{figure}
\subsubsection*{Strahlungskorrekturen}
\subsection{Vorwärts-Rückwärts Asymmetrie}
\subsection{Zerfallsbreiten}
\label{sub:theo:gamma}
Die Zerfallsbreite $\Gamma$ ist durch die Energie-Zeit-Unschärfe motiviert und hängt folgendermaßen mit der Lebensdauer $\tau$ zusammen:
\begin{equation}
    \Gamma = \frac{\hbar}{\tau}
\end{equation}
Betrachtet man die Zerfallsbreite eines Zerfallskanals, so spricht man von der \emph{partielle} Zerfallsbreite. Die \emph{totale} Zerfallsbreite 
setzt sich aus der Summe aller partiellen Zerfallsbreiten zusammen. \\
Für die (totale) Zerfallsbreite des \Z-Bosons gilt \cite{manual}:
\begin{equation}
    \Gamma_\text{Z} = \Gamma_\text{e} + \Gamma_\text{\textmu} + \Gamma_\text{\texttau} + \Gamma_\text{q} + n \cdot \Gamma_\nu + \Gamma_\text{unbek.}
\end{equation}
Die partiellen Zerfallsbreiten der verschiedenen Quark-Zerfallskanäle wurden zu $\Gamma_\text{q}$ zusammengefasst. Die Zerfallsbreite der 
nicht im Standardmodell vorhergesagten Zerfallskanäle wurde mit $\Gamma_\text{unbek.}$ bezeichnet. Allerdings wurden bis heute keine solchen 
Prozesse beobachtet, weshalb $\Gamma_\text{unbek.}$ Null gesetzt werden kann.\\
$\Gamma_\nu$ ist die Zerfallsbreite einer leichten Neutrino-Generation. Sind alle Zerfallsbreiten bekannt, so kann die Anzahl $n$ der leichten 
Neutrino-Generationen vorhergesagt werden.

\subsection{Theoretische Berechnung der Zerfallsbreiten und Wirkungsquerschnitte}
\subsubsection*{Zerfallsbreiten}
\subsubsection{Wirkungsquerschnitte}
Der Wirkungsquerschnitt am \Z-Resonanzmaximum ($s = M_\text{Z}$) lässt sich nach \autoref{eq:sigma:fermion} folgendermaßen berechnen:
\begin{equation}
    \sigma_f^\text{peak} = \frac{12 \pi}{M_\text{Z}^2} \frac{\Gamma_\text{e} \Gamma_f}{\Gamma_\text{Z}^2}
\end{equation}