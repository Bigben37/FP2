\section{Physikalische Grundlagen}
\subsection{Das Standardmodell}
\subsubsection*{Teilchen im Standardmodell}
Das Standardmodell der Teilchenphysik mit seinen 19 (oder mehr, mit Neutrinooszillation) freien Parametern
bildet den theoretischen Rahmen für das durchgeführte Experiment.
Im folgenden werden kurz die Elementarteilchen beschrieben, die im Standardmodell vorkommen (\autoref{img:standardmodel}).
Sie lassen sich in drei Gruppen einteilen:
\emph{Fermionen}, \emph{Eichbosonen} und das \emph{Higgs-Teilchen}.
Die Eichbosonen vermitteln drei fundamentale Wechselwirkungen zwischen den Teilchen:
Das Photon die elektromagnetische Kraft, das Gluon die starke Kraft und W- und Z-Bosonen die schwache Kraft.
Die Gruppe der Fermionen besteht aus 6 Quarks, 3 geladenen Leptonen und 3 Neutrinos.
Je vier der Teilchen werden zu einer Generation zusammengefasst.

\begin{figure}[H]
    \centering
    \def\svgwidth{0.55\textwidth}
    \input{../img/standardmodel.pdf_tex}
    \caption{Die Teilchen im Standardmodell.}
    \label{img:standardmodel}
\end{figure}

\subsubsection*{Elektroschwache Wechselwirkung}
Die elektromagnetische und die schwache Wechselwirkung wurden 1967 von Glashow, Salam und Weinberg zu der elektroschwachen Theorie vereinheitlicht,
da sich die Kopplungskonstanten der elektromagnetischen und schwachen Wechselwirkung bei hohen Energien im Wert ähneln.
In der elektroschwachen
Theorie gibt es vier Eichbosonen, die keine Masse haben\footnote{Die Masse erhalten drei der vier Eichbosonen durch den \emph{Higgs-Kibble-Mechanismus}.}:
$\text{W}^\pm$ und $\text{W}^0$ aus der schwachen Wechselwirkung und $\text{B}^0$
aus der elektromagnetischen Wechselwirkung. Man würde intuitiv das $\text{B}^0$ mit dem Photon gleichsetzten, jedoch koppelt das $\text{B}^0$
auch an Neutrinos, das Photon aber nicht. \\
Das Photonfeld wird als Überlagerung von $\text{W}^0$ und $\text{B}^0$ definiert
mit dem \emph{Weinbergwinkel} $\theta_\text{W}$ als Mischungswinkel der beiden Felder. Das zum Photonfeld orthogonale Feld ist das Feld des \Z.
Der Weinbergwinkel lässt sich mit
\begin{equation}
    \cos \theta_\text{W} = \frac{M_\text{W}}{M_\text{Z}} \qquad \text{oder} \qquad \sin^2 \theta_\text{W} = \frac{\alpha_\text{em}}{\alpha_\text{W}}
\end{equation}
bestimmen \cite{manual}. Dabei sind $M_\text{Z}$ und $M_\text{W}$ die Massen des \Z- bzw. W-Bosons, $\alpha_\text{em}$ die Kopplungskonstante
der elektromagnetischen Wechselwirkung und $\alpha_\text{W}$ die Kopplungskonstante der schwachen Wechselwirkung.

\subsection{Der Wirkungsquerschnitt}
In der Teilchenphysik gibt der Wirkungsquerschnitt $\sigma$ die Wahrscheinlichkeit an, mit der ein gewisser Prozess bei der Kollision von 
Teilchen auftritt.
Die Einheit des Wirkungsquerschnitts ist \emph{barn} ($1\,\text{b} = 10^{-24}\,\text{cm}^2$). \\
Jedoch wird im Experiment nicht direkt der Wirkungsquerschnitt gemessen, sondern die Anzahl eines gewissen Ereignisses (Zerfallskanal). Mit Hilfe
der \emph{Luminosität} $L$, ein Maß für die Anzahl der Kollisionen pro Fläche und Zeit, kann der Wirkungsquerschnitt bestimmt werden:
\begin{equation}
    \sigma = \frac{N}{\int L \, \difd t}
\end{equation}
Dabei ist $\int L \, \difd t$ die (über die Zeit) integrierte Luminosität. \\
Wird der Wirkungsquerschnitt eines Zerfallskanals gegen die Schwerpunktsenergie $\sqrt{s}$ der kollidierenden Teilchen aufgetragen, so ist
bei manchen Energien eine deutliche Erhöhung des Wirkungsquerschnitts zu erkennen. Dieser Effekt wird Resonanz genannt und wird durch das Teilchen
verursacht, welches die Energie besitzt, bei der die Erhöhung des Wirkungsquerschnitts stattfindet.
So gibt es zum Beispiel bei ungefähr 91\,GeV einen Resonanzpeak, der dem \Z-Boson zugeordnet wird. \\
Der Verlauf der  \Z-Resonanzkurve eines fermionischen Zerfallskanals kann mit einer relativistischen Breit-Wigner Kurve (analog zur Resonanz des
klassischen harmonischen Oszillators) beschrieben werden \cite{manual}:
\begin{equation}
    \label{eq:sigma:fermion}
    \sigma_f(s) = \frac{12 \pi}{M_\text{Z}^2} \cdot \frac{s \Gamma_\text{e} \Gamma_f}{ \left( s - M_\text{Z}^2 \right)^2 + \left( s^2 \Gamma_\text{Z}^2 / M_\text{Z}^2 \right) }
\end{equation}
$M_\text{Z}$ ist die Masse des \Z-Bosons, $\Gamma_\text{e}$ die elektronische Zerfallsbreite und $\Gamma_f$ die Zerfallsbreite des Fermions. Auf
die Zerfallsbreiten wird genauer in \autoref{sub:theo:gamma} eingegangen.
\subsection{\texorpdfstring{\elp-\elm}{e+/e-}-Wechselwirkung}
\subsubsection*{Zerfallskanäle}
Im LEP (siehe \autoref{sec:setup}) werden Kollisionen von Elektronen und Positronen untersucht.
Vernichten sich \elp\ und \elm\ (s-Kanal\footnote{Die Bezeichnungen s-Kanal (Annihilation) und t-Kanal (Streuung) gehen auf die
\emph{Mandelstam-Variablen} zurück.}), so kann entweder ein Photon oder ein \Z-Boson entstehen, welche beide nach kurzer Zeit wieder
zerfallen. Bei einer Schwerpunktsenergie $\sqrt{s}$ in der Nähe der \Z-Masse ist die Produktion von Photonen jedoch stark unterdrückt,
weshalb diese Zerfallskanäle nicht relevant für die Bestimmung von Eigenschaften des \Z-Bosons sind. \\
Das \Z-Boson kann in $\ell^+\ell^-$-, $\nu_\ell\bar{\nu}_\ell$- und \qq-Paare zerfallen (\autoref{img:Z0:decay}).
Dabei bezeichnet $\ell$ eines der drei Leptonen (e, \textmu, \texttau), $\nu_\ell$ ihre Neutrinos und q eines der fünf leichtesten Quarks 
(u, d, c, s, b). Ein t$\bar{\text{t}}$-Paar
kann nicht erzeugt werden, da die Masse des top-Quarks größer ist als die von \Z.
\begin{figure}[H]
    \centering
    \def\svgwidth{0.7\textwidth}
    \input{../img/Z0decay.pdf_tex}
    \caption{Zerfallskanäle des \Z-Bosons.}
    \label{img:Z0:decay}
\end{figure}
\subsubsection*{Bhabha-Streuung}
\label{subsub:theo:stchannel}
Neben der Annihilation von \ee\ gibt es auch die \ee-Streuung
(der t-Kanal der \emph{Bha\-bha-Streu\-ung}, \autoref{img:bhabha}).
Die Bhabha-Streuung wird durch die Reaktion \ee$\to$\ee\ beschrieben.
Da für die experimentelle Bestimmung der \Z-Breite nur der s-Kanal relevant ist, müssen die Ereignisse des t-Kanals gefiltert werden.
Dies ist durch eine Unterscheidung der Zerfallsprodukte nicht möglich, weil sie in beiden Kanälen gleich sind. Allerdings besitzen
die Wirkungsquerschnitte von s- und t-Kanal eine unterschiedliche Winkelabhängigkeit des Polarwinkels $\Theta$
(die Strahlachse entspricht der $z$-Achse). Es gilt~\cite{manual}:
\begin{equation}
    \label{eq:stchannel:sigmas}
    \frac{\difd \sigma_\text{s}}{\difd \Omega} \sim \left( 1 + \cos^2 \Theta \right), \qquad
    \frac{\difd \sigma_\text{t}}{\difd \Omega} \sim \left( 1 - \cos \Theta \right)^{-2}
\end{equation}
\begin{figure}[H]
    \centering
    \def\svgwidth{0.5\textwidth}
    \input{../img/tchanBhabha.pdf_tex}
    \caption{Der t-Kanal der Bhabha-Streuung.}
    \label{img:bhabha}
\end{figure}

\subsubsection*{Strahlungskorrekturen}
\label{sect:strahlungskorr}
Um die Messergebnisse bei den im Experiment verwendeten hohen Energien korrekt beschreiben zu können,
reicht die Bornsche Näherung (Berücksichtigung der Feynman-Diagramme bis zur ersten Ordnung) nicht mehr aus.
An den Beispielen der Anfangsbremsstrahlung und der Endbremsstrahlung wird hier
eine reelle Strahlungskorrektur beschrieben.
Zur Bestimmung der Korrekturfaktoren müssen zusätzlich noch virtuelle Korrekturen (Schleifen mit gleichen Endzuständen 
wie im Diagramm erster Ordnung) und Gluonabstrahlung berücksichtigt werden.

Wenn die Schwerpunktsenergie von Elektron und Positron im Bereich der \Z-Resonanz liegt,
und ein Reaktionspartner vor der Wechselwirkung ein Photon abstrahlt (\autoref{img:radkorr} links),
so verringert sich die zur Verfügung stehende Energie und der Wirkungsquerschnitt nimmt ab.
Ist Energie der beiden Partner aber größer als die Masse des \Z, dann erhöht die Abstrahlung
eines Photons die Wahrscheinlichkeit der Bildung eines \Z\ und der Wirkungsquerschnitt nimmt zu,
weil die Energie der Reaktionspartner dann
näher an der Resonanz liegt.

\begin{figure}[H]
    \centering
    \def\svgwidth{0.75\textwidth}
    \input{../img/radkorr.pdf_tex}
    \caption{Feynman-Diagramme für die Strahlungskorrektur durch Anfangs- und Endbremsstrahlung.}
    \label{img:radkorr}
\end{figure}

\subsection{Vorwärts-Rückwärts-Asymmetrie}
Unter der Vorwärts-Rückwärts-Asymmetrie versteht man die relative Differenz der Wirkungsquerschnitte in vorderer und hinterer Hemisphäre:
\begin{equation}
    \label{eq:theo:FBA:deflong}
    \begin{split}
        & \sigma_\text{F} = \int_0^1 \frac{\difd \sigma_f}{\difd \cos \Theta} \difd \cos \Theta \\
        & \sigma_\text{B} = \int_{-1}^0 \frac{\difd \sigma_f}{\difd \cos \Theta} \difd \cos \Theta \\
        & A_\text{FB}^f = \frac{\sigma_\text{F} - \sigma_\text{B}}{\sigma_\text{F} + \sigma_\text{B}}
    \end{split}
\end{equation}
$\sigma_\text{F}$ und $\sigma_\text{B}$ sind die Wirkungsquerschnitte in Vorwärts- bzw. Rückwärtshemisphäre und $A_\text{FB}^f$ die
Vorwärts-Rückwärts-Asymmetrie für das Fermion $f$. \\
Der differentielle Wirkungsquerschnitt $\frac{\difd \sigma_f}{\difd \cos \Theta}$ kann mit
\begin{equation}
    \frac{\difd \sigma_f}{\difd \cos \Theta} = A(s) \left[F_1(s) \left( 1 + \cos^2 \Theta \right) + 2 F_2(s) \cos \Theta \right]
\end{equation}
genähert werden \cite{manual}. Dabei sind $A(s)$, $F_1(s)$ und $F_2(s)$ Faktoren,
deren Abhängigkeit von der Schwerpunktsenergie $\sqrt{s}$ für diesen Versuch nicht wichtig ist. \\
Nun lässt sich die Vorwärts-Rückwärts-Asymmetrie durch Lösen der Integrale in \autoref{eq:theo:FBA:deflong} zu
\begin{equation}
    A_\text{FB}^f = \frac{3}{4} \frac{F_2}{F_1}
\end{equation}
berechnen. Für Leptonen gilt an der Spitze der Resonanzkurve \cite{manual}:
\begin{equation}
    A_\text{FB}^{\ell, \text{peak}} \simeq 3 \left( \frac{g_\text{V}^\ell}{g_\text{A}^\ell} \right)^2 = 3 \left( 1 - 4 \sin^2 \theta_\text{W} \right)^2
\end{equation}
Somit kann man den Weinbergwinkel $\theta_\text{W}$ bestimmen, wenn man die Vorwärts-Rückwärts-Asymmetrie kennt.
\subsection{Zerfallsbreiten}
\label{sub:theo:gamma}
Die Zerfallsbreite $\Gamma$ ist durch die Energie-Zeit-Unschärfe motiviert und hängt folgendermaßen mit der Lebensdauer $\tau$ zusammen:
\begin{equation}
    \Gamma = \frac{\hbar}{\tau}
\end{equation}
Betrachtet man die Zerfallsbreite eines Zerfallskanals, so spricht man von der \emph{partiellen} Zerfallsbreite. Die \emph{totale} Zerfallsbreite
setzt sich aus der Summe aller partiellen Zerfallsbreiten zusammen. \\
Für die (totale) Zerfallsbreite des \Z-Bosons gilt \cite{manual}:
\begin{equation}
    \Gamma_\text{Z} = \Gamma_\text{e} + \Gamma_\text{\textmu} + \Gamma_\text{\texttau} + \Gamma_\text{q} + n \cdot \Gamma_\nu + \Gamma_\text{unbek.}
\end{equation}
Die partiellen Zerfallsbreiten der verschiedenen Quark-Zerfallskanäle wurden zu $\Gamma_\text{q}$ zusammengefasst. Die Zerfallsbreite der
nicht im Standardmodell vorhergesagten Zerfallskanäle wurde mit $\Gamma_\text{unbek.}$ bezeichnet. Allerdings wurden bis heute keine solchen
Prozesse beobachtet, weshalb $\Gamma_\text{unbek.}$ gleich Null gesetzt werden kann.\\
$\Gamma_\nu$ ist die Zerfallsbreite einer leichten Neutrino-Generation. Sind alle Zerfallsbreiten bekannt, so kann die Anzahl $n$ der leichten
Neutrino-Generationen vorhergesagt werden.

\subsection{Theoretische Berechnung der Zerfallsbreiten und Wirkungsquerschnitte}
\subsubsection*{Zerfallsbreiten}
Alle Formeln und Konstanten stammen, wenn nicht anders angegeben, aus \cite{manualmuc}. \\
Die partielle Breite der \Z-Resonanz berechnet sich für einen fermionischen Zerfallskanal mit:
\begin{equation}
    \label{eq:theo:calc:gamma}
    \Gamma_f = \frac{N_c^f \cdot \sqrt{2}}{12 \pi} \cdot G_F \cdot M_Z^3 \cdot \left( \left( g_\text{V}^f \right)^2 + \left( g_\text{A}^f \right)^2 \right)
\end{equation}
Dabei sind schwachen Kopplungen (Vektor und Axialvektor) durch
\begin{equation}
    g_\text{V}^f = I_3^f - 2 Q_f \sin^2 \theta_W, \qquad g_A^f = I_3^f
\end{equation}
gegeben. Einsetzen in \autoref{eq:theo:calc:gamma} liefert:
\begin{equation}
    \Gamma_f = \frac{N_c^f \cdot \sqrt{2}}{12 \pi} \cdot G_F \cdot M_Z^3 \cdot \left( \left( I_3^f - 2 Q_f \sin^2 \theta_W \right)^2 + \left( I_3^f \right)^2 \right)
\end{equation}
Die zur Berechnung benötigten Konstanten sind:
\begin{itemize}
    \item Farbfaktor
    \begin{equation}
        N_c^f =
        \begin{cases}
            1 & \text{für Leptonen } (f = \text{e}, \text{\textmu}, \text{\texttau}, \nu) \\
            3 (1 + \delta_\text{QCD}) & \text{für Quarks } (f = \text{u,\,d,\,c,\,s,\,b})
        \end{cases}
    \end{equation}
    mit dem QCD-Korrekturterm $\delta_\text{QCD} = 1.05 \frac{\alpha_\text{S}(M_\text{Z})}{\pi}$ und der starken Kopplungskonstante
    $\alpha_\text{S}(M_\text{Z}) = 0.12$ bei einer Energie von 92.1\,GeV.
    \item Fermikonstante $G_F = 1.1663 \cdot 10^{-5}\,\text{GeV}^{-2}$ \cite{manual}
    \item Masse des \Z-Bosons $M_\text{Z}=91.187$\,GeV
    \item $z$-Komponente des schwachen Isospins
    \begin{equation}
        I_3^f =
        \begin{cases}
            \frac{1}{2}  & \text{für } f = \nu, \text{u,\,c} \\
            -\frac{1}{2} & \text{für } f = \text{\elm,\,\textmu$^-$,\,\texttau$^-$,\,d,\,s,\,b}
        \end{cases}
    \end{equation}
    \item Ladung
    \begin{equation}
        Q_f =
        \begin{cases}
            0			& \text{für } f = \nu \\
         -1			& \text{für } f = \text{\elm,\,\textmu$^-$,\,\texttau$^-$} \\
      \frac{2}{3}	& \text{für } f = \text{u,\,c} \\
     -\frac{1}{3}& \text{für } f = \text{d,\,s,\,b}
        \end{cases}
    \end{equation}
    \item Weinbergwinkel $\theta_\text{W}$
    \begin{equation}
        \sin^2 \theta_\text{W} = \frac{1}{2} - \sqrt{\frac{1}{4} - \frac{\pi \cdot \alpha \left( M_\text{Z}^2 \right) }{\sqrt{2} \cdot G_\text{F} \cdot M_\text{Z}^2}}
    \end{equation}
    Dabei ist $\alpha \left( M_\text{Z}^2 \right) = \frac{1}{128.87}$ die elektromagnetische Kopplungskonstante bei 91.2\,GeV. Der Weinbergwinkel
    berechent sich zu
    \begin{equation}
        \sin^2 \theta_\text{W} = 0.231241 \ \, .
    \end{equation}
\end{itemize}
Die berechneten partiellen Breiten
der \Z-Resonanz sind in \autoref{tab:theo:calc:gammas} aufgelistet.
Abweichungen der theoretischen Werte von den Literaturwerten treten auf,
weil bei ihrer Berechnung nur die Feynman-Diagramme erster Ordnung berücksichtigt wurden.
\begin{table}[H]
\caption{Berechnete partielle Breiten $\Gamma_f$ und Literaturwerte der \Z-Resonanz.}
\begin{center}
\begin{tabular}{|c|c|c|}
    \hline
    $f$ 					& $\Gamma_f$ / MeV 	& $\Gamma_f^\text{Lit.}$ / MeV	\\ \hline\hline
    e, \textmu, \texttau	& 83.3873			& 83.984 \pm\ 0.086\ \cite{pdg}	\\ \hline
    e + \textmu + \texttau	& 250.162			& - 							\\ \hline
    $\nu$					& 165.841			& 167.6	\cite{manual}			\\ \hline
    3$\nu$					& 497.523			& 499.0 \pm\ 1.5\ \cite{pdg}		\\ \hline
    u, c					& 296.763			& 299 \cite{manual}				\\ \hline
    d, s, b					& 382.524			& 378 \cite{manual}				\\ \hline
    u + c + d  + s + b		& 1741.1			& 1744.4 \pm\ 2.0\ \cite{pdg}	\\ \hline
\end{tabular}
\end{center}
\label{tab:theo:calc:gammas}
\end{table}
Als Gesamtbreite $\Gamma_\text{Z}$ erhält man die Summe aus leptonischen, hadronischen und Neutrino-Zerfällen
\begin{equation}
  \Gamma_\text{Z} =  2488.78\, \text{MeV}\ \, .
\end{equation}
Der Literaturwert \cite{pdg} beträgt hier
\begin{equation}
  \Gamma_\text{Z}^\text{Lit.} =  (2495.2 \pm 2.3)\, \text{MeV}\ \, .
\end{equation}
Der Grund für die Abweichung um fast drei Standardabweichungen wurde oben erwähnt. \\
Geht man von einer zusätzlichen Leptonenfamilie aus, so ändert sich die Gesamtbreite um
\begin{equation}
    \frac{\Gamma_\text{e} + \Gamma_\nu}{\Gamma_\text{Z}} \approx 10.0\% \ \, .
\end{equation}


\subsubsection{Wirkungsquerschnitte}
Die partiellen Wirkungsquerschnitte am \Z-Resonanzmaximum ($\sqrt{s} = M_\text{Z}$) lassen sich nach \autoref{eq:sigma:fermion} folgendermaßen berechnen:
\begin{equation}
    \sigma_f^\text{peak} = \frac{12 \pi}{M_\text{Z}^2} \frac{\Gamma_\text{e} \Gamma_f}{\Gamma_\text{Z}^2}
\end{equation}
Die Werte für $\Gamma_f$, $\Gamma_\text{e}$ und $M_\text{Z}$ wurden aus dem letzten Abschnitt übernommen. Die berechneten Werte sind in 
\autoref{tab:theo:calc:sigmas} aufgeführt.
\begin{table}[H]
    \caption{Berechnete partielle Wirkungsquerschnitte $\sigma_f$ am Peak der \Z-Resonanz.}
    \begin{center}
        \begin{tabular}{|c|c|}
            \hline
            $f$ 					& $\Gamma_f$ / nB 	\\ \hline
      		e, \textmu, \texttau	& 1.98214			\\ \hline
  			e + \textmu + \texttau	& 5.94642			\\ \hline
			$\nu$					& 3.94209			\\ \hline
			3$\nu$					& 11.8263			\\ \hline
			u, c					& 7.05416			\\ \hline
			d, s, b					& 9.09271			\\ \hline
			u + c + d  + s + b		& 41.3864			\\ \hline
        \end{tabular}
    \end{center}
    \label{tab:theo:calc:sigmas}
\end{table}
Der totale Wirkungsquerschnitt ergibt sich zu
\begin{equation}
    \sigma_\text{tot} = 59.1591\,\text{nB} \, .
\end{equation}