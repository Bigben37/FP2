\section{Physikalische Grundlagen}
\subsection{Standardmodell}
\subsubsection{Teilchen}
\subsubsection{Kräfte}
\paragraph{Gravitation}
\paragraph{Starke Wechselwirkung}
\paragraph{Elektroschwache Wechselwirkung}
\subsection{Wirkungsquerschnitt}
Klassisch gesehen beschreibt der Wirkungsquerschnitt $\sigma$ bei einer Kollision von zwei Objekten die Fläche, an der sie sich treffen. 
In der Teilchenphysik gibt der Wirkungsquerschnitt eine Wahrscheinlichkeit, ob ein gewisser Prozess bei der Kollision von Teilchen auftritt, an.
Die Einheit des Wirkungsquerschnitts ist \emph{barn} ($1\,\text{b} = 10^{-24}\,\text{cm}^2$). \\
Jedoch wird im Experiment nicht direkt der Wirkungsquerschnitt gemessen, sondern die Anzahl eines gewissen Ereignisses (Zerfallskanal). Mit Hilfe 
der \emph{Luminosität} $L$, ein Maß für die Anzahl der Kollisionen pro Fläche und Zeit, kann der Wirkungsquerschnitt bestimmt werden:
\begin{equation}
    \sigma = \frac{N}{\int L \, \difd t}
\end{equation}
\subsubsection{Wirkungsquerschnitt }
\subsection{\texorpdfstring{\elp-\elm}{e+/e-}-Wechselwirkung}
\subsubsection{Zerfallskanäle}
Im LEP (siehe \autoref{sec:setup}) werden Kollisionen von Elektronen und Positronen untersucht.
Vernichten sich \elp und \elm\ (s-Kanal\footnote{Die Bezeichnungen s-Kanal (Annihilation) und t-Kanal (Streuung) gehen auf die 
\emph{Mandelstammvariablen} zurück.}), so kann entweder ein Photon oder ein \Z-Boson entstehen, welche beide nach kurzer Zeit wieder 
zerfallen. Bei einer Schwerpunktsenergie $\sqrt{s}$ in der Nähe der \Z-Masse ist die Produktion von Photonen jedoch stark unterdrückt, 
weshalb diese Zerfallskanäle nicht relevant für die Bestimmung von Eigenschaften des \Z-Bosons sind. \\
Das \Z-Boson kann in $\ell^+\ell^-$-, $\nu_\ell\bar{\nu}_\ell$- und $q\bar{q}$-Paare zerfallen (\autoref{img:Z0:decay}).
Dabei bezeichnet $\ell$ eines der drei Leptonen (e, \textmu, \texttau) und $q$ eines der fünf leichtesten Quarks (u, d, c, s, b). Ein \xx{t}-Paar 
kann nicht erzeugt werden, da die Masse des top-Quarks größer ist als die von \Z.
\subsubsection{Bhabha-Streuung}
Neben der Annihilation von \ee gibt es auch die \ee-Streuung (der t-Kanal der \emph{Bha\-bha-Streu\-ung}, \autoref{img:bhabha}).
Die Bhabha-Streuung wird durch die Reaktion \ee$\to$\ee beschrieben.
Da für die experimentelle Bestimmung der \Z-Breite nur der s-Kanal relevant ist, müssen die Ereignisse des t-Kanals gefiltert werden. 
Dies ist durch eine Unterscheidung der Zerfallsprodukte nicht möglich, da sie in beiden Kanälen gleich sind. Allerdings besitzten 
der Wirkungsquerschnitte von s- und t-Kanal eine unterschiedliche Winkelabhängigkeit des Polarwinkels $\Theta$ 
(die Strahlachse entspricht der $z$-Achse). Es gilt \cite{manual}:
\begin{equation}
    \label{eq:stchannel:sigmas}
    \frac{\difd \sigma_\text{s}}{\difd \Omega} \sim \left( 1 + \cos^2 \Theta \right), \qquad 
    \frac{\difd \sigma_\text{t}}{\difd \Omega} \sim \left( 1 - \cos \Theta \right)^{-2}
\end{equation}
\subsubsection{Strahlungskorrekturen}
\subsection{Vorwärts-Rückwärts Asymmetrie}
\subsection{Zerfallsbreiten}
\subsection{Theoretische Berechnung der Zerfallsbreiten}