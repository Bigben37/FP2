\section{Messergebnisse und Auswertung}
\subsection{Bestimmung der Schnitte}
Reinheiten
\subsection{Berechnung der Effizienzmatrix}
Werden die oben bestimmten Schnitte auf die Monte-Carlo-Daten angewendet, kann man die Effizienz $\bm{E}$ der Schnitte berechnen. 
Die Effizienzmatrix gibt an, welcher Anteil der verschiedenen Ereignisse nach einem Schnitt noch vorhanden sind. Sie ist definiert als 
\begin{equation}
    \left( \bm{E} \right)_{i,j} = \frac{n_{i, j}^\text{cut}}{n_i} \ \, .
\end{equation} 
$i \in $ e$^+$e$^-$, \textmu$^+$\textmu$^-$, \texttau$^+$\texttau$^-$, q$^+$q$^-$ \\  %TODO format
Dabei bezeichnet $n_{i, j}^\text{cut}$ die Anzahl der Ereignisse von Typ $i$ nach Schnitt von Typ $j$ und $n_i$ die gesamte Anzahl von Ereignissen 
der Monte-Carlo-Simulation von Typ $i$. 
Die Beziehung zwischen den Anzahlen von gemessenen Ereignissen nach Schnitt $\vec{M}$\footnote{"measurement"} und den echten Anzahlen 
$\vec{T}$\footnote{"truth"} lautet durch die Definition der Effizienzmatrix $\bm{E}$:
\begin{equation}
    \label{eq:effmat:mtrel}
    \vec{M} = \bm{E} \vec{T}
\end{equation}
Die Effizienzmatrix unsere Schnitte ist in \autoref{tab:effmat:val} dargestellt.
\begin{table}[H]
\caption{Effizienzmatrix.}
\begin{center}
\begin{tabular}{|c|c|c|c|c|}
  \hline
  Schnitt$\backslash$MC-Daten & \ee & \mm & \tt & \qq \\ \hline
  \ee & 0.388233 & 0.000011 & 0.001995 & 0.000010 \\ \hline
  \mm & 0.000160 & 0.890762 & 0.003585 & 0.000000 \\ \hline
  \tt & 0.001791 & 0.004259 & 0.747406 & 0.002354 \\ \hline
  \qq & 0.000000 & 0.000000 & 0.001868 & 0.966965 \\ \hline
\end{tabular}
\end{center}
\label{tab:effmat:val}
\end{table}

Nun gilt es den Fehler der einzelnen Einträge der Effizienzmatrix zu bestimmen. Hierzu wird die Definition der Effizienz $\epsilon$ leicht geändert:
\begin{equation}
    \epsilon = \frac{p}{p+f}	
\end{equation}
Die Anzahl der Ereignisse, die nach Schnitt noch da sind, wird mit $p$\footnote{"`pass"'} bezeichnet. Aus der totalen Anzahl $n$ der Ereignisse lässt 
sich die Anzahl derjenigen Ereignisse $f$\footnote{"`fail"'} berechnen, die beim Schnitt wegfallen. Da $p$ und $f$ als poissonverteilt angenommen 
werden können (bei hinreichenden Größen von $p$ und $f$), gilt für den Fehler der Effizienz mit Gauß'scher Fehlerfortpflanzung.
\begin{equation}
    s_\epsilon = \sqrt{\frac{f \cdot p}{ \left( f + p \right)^3}}
\end{equation}
Mit der Definition von $\epsilon$ und $n$ kann man den Fehler zu der in der Literatur üblichen Form umformen:
\begin{equation}
    s_\epsilon = \sqrt{\frac{\epsilon (1-\epsilon)}{n}}
\end{equation}
Somit können nun die Fehler von den Einträge der Effizienzmatrix bestimmt werden (\autoref{tab:effmat:err}).
\begin{table}[H]
\caption{Fehler der Effizienzmatrix.}
\begin{center}
\begin{tabular}{|c|c|c|c|c|}
  \hline
  Schnitt$\backslash$MC-Daten & \ee & \mm & \tt & \qq \\ \hline
  \ee & 0.001591 & 0.000011 & 0.000159 & 0.000010 \\ \hline
  \mm & 0.000041 & 0.001015 & 0.000212 & 0.000000 \\ \hline
  \tt & 0.000138 & 0.000212 & 0.001544 & 0.000154 \\ \hline
  \qq & 0.000000 & 0.000000 & 0.000153 & 0.000569 \\ \hline
\end{tabular}
\end{center}
\label{tab:effmat:err}
\end{table}

\subsubsection{Berechnung der inversen Effizienzmatrix}

\subsection{s-t-Kanal Trennung}
\subsection{Auswertung der Wirkungsquerschnitte}
Fits
\subsubsection{Masse}
\subsubsection{Totale Zerfallsbreite}
\subsubsection{Partielle Zerfallsbreiten}
Leptonenuniversalität
\subsubsection{Anzahl leichter Neutrinogenerationen}
\subsection{Vorwärts-Rückwärts Assymetrie}

Zitat BRs \cite{pdg}.